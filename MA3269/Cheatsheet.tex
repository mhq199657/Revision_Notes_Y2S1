\PassOptionsToPackage{svgnames}{xcolor}
\documentclass[11pt]{article}



\usepackage[margin=0.1in]{geometry}  
\usepackage{graphicx}             
\usepackage{amsmath}              
\usepackage{amsfonts}              
\usepackage{framed}               
\usepackage{amssymb}
\usepackage{array}
\usepackage{amsthm}
\usepackage[nottoc]{tocbibind}
\usepackage{bm}
\usepackage{enumitem}


  \newcommand\norm[1]{\left\lVert#1\right\rVert}
\setlength{\parindent}{0cm}
\setlength{\parskip}{0em}
\newcommand{\Lim}[1]{\raisebox{0.5ex}{\scalebox{0.8}{$\displaystyle \lim_{#1}\;$}}}
\newtheorem{definition}{Definition}[section]
\newtheorem{theorem}{Theorem}[section]
\newtheorem{notation}{Notation}[section]
\theoremstyle{definition}
\DeclareMathOperator{\arcsec}{arcsec}
\DeclareMathOperator{\arccot}{arccot}
\DeclareMathOperator{\arccsc}{arccsc}
\DeclareMathOperator{\PV}{PV}
\DeclareMathOperator{\TV}{TV}
\DeclareMathOperator{\diff}{d}
\DeclareMathOperator{\expec}{E}
\DeclareMathOperator{\CE}{CE}
\DeclareMathOperator{\RP}{RP}
\newcommand\cf[1]{\mathbf{#1}}
\setcounter{tocdepth}{1}
\begin{document}
\twocolumn
\section{Theory of Interest}
\subsection{Interest}
\begin{definition}[Accumulation Function]
\hfill\\\normalfont When a principal of 1 dollar is deposited in an interest-paying account at time $t=0$, it earns some interest over the time interval $[0,t]$. \\
The accumulated value of 1 dollar at time $t \geq 0$, denoted by $a(t)$, is known as the \textbf{accumulation function}. Clearly, $a(0)=1$.
\end{definition}
\begin{definition}[Simple and Compound Interest]
\hfill\\\normalfont Let $r$ be the annual rate of interest.\\
Based on the \textbf{simple-interest} method of calculating interest,
\[
a(t)=1+rt\;\;\;\text{for }t \geq 0
\]
If the \textbf{compound interest} method is used,
\[
a(t)=(1+r)^t \;\;\;\text{for }t\geq 0
\]
\end{definition}
Suppose the interest rate is $r_i$ for the period $[\sum_{k=0}^{i-1} t_i,\sum_{k=1}^{i} t_{i} ]$, where $t_0 = 0$, 
\[
a(t_j)=1+\sum_{i=1}^j r_it_i \;\;\;\text{when simple interest is used;}
\]
\[
a(t_j)=\prod_{i=1}^j (1+r_i)^{t_i}\;\;\;\text{when compound interest is used;}
\]
\begin{definition}[Frequency of Compounding]
\hfill\\\normalfont When an interest of $r = r^{(p)}$ is paid $p$ times a year (or equivalently, $r^{(p)}$ is \textbf{convertible} $p$\textbf{thly} or $r^{(p)}$ is compounded $p$ times a year), we call $p$ the \textbf{frequency of compounding} and $r^{(p)}$ the \textbf{nominal} rate of interest.
\end{definition}
The interest to be paid over the period, is $\frac{r^{(p)}}{p}$. Effectively, \$1 invested at time $t=0$ will grow to $\left(1+\frac{r^{(p)}}{p}\right)$ over a period of length $\frac{1}{p}$, so that the accumulated amount after one year is$\left(1+\frac{r^{(p)}}{p}\right)^p$.
\textbf{Remarks}
\begin{enumerate}
  \item We write the superscript $(p)$ for $r^{(p)}$ to indicate the frequency of compounding $p$.
  \item We can drop the superscript $(p)$ when $p=1$.
  \item $p = 2, 4, 12 $ correspond to semi-annual, quarterly and monthly compounding respectively,
\end{enumerate}
\begin{definition}[Equivalent Interest Rates]
\hfill\\\normalfont Two nominal interest rates are said to be \textbf{equivalent} if and only if they yield same accumulation amount over a year. Hence, the nominal rates $r^{(p)}$ and $r^{(q)}$ are equivalent if and only if
\[
\left(1+\frac{r^{(p)}}{p}\right)^p=\left(1+\frac{r^{(q)}}{q}\right)^q
\]
In particular, the \textbf{effective} annual interest rate (when $p=1$), denoted by $r_e$, is given by
\[
1+r_e = \left(1+\frac{r^{(p)}}{p}\right)^p
\]
The corresponding accumulation function is
\[
a(t) = (1+r_e)^t = \left(1+\frac{r^{(p)}}{p}\right)^{pt}
\]
\end{definition}
It can be shown that $r_e\geq r^{(p)}$ for $p>1$.
\begin{definition}[Continuous Compounding]
\hfill\\\normalfont The interest is \textbf{compounded continuously} when the frequency of compounding tends to infinity.\\
Let $r^{(\infty)}$ denote the nominal rate of interest under continuous compounding. Then,
\[
a(1)=\Lim{p\to\infty}\left(1+\frac{r^{(\infty)}}{p}\right)^p = e^{r^{(\infty)}}
\]
\end{definition}
The number $r^{(\infty)}$ is known as the \textbf{continuously compounded} rate of interest.
The corresponding accumulatio function is
\[
a(t) = e^{r^{(\infty)}t},\;\;\;t\geq 0
\]
Note that $e^{r^{(\infty)}}= 1+r_e$.\\
It can be shown that
\[
e^r>\left(1+\frac{r}{p}\right)^p
\]
for any $r>0$ and for any $p\in\mathbb{Z}^+$.
\subsection{Present Value}
\begin{definition}[Present Value, Time Value]
\hfill\\\normalfont Let $a(t)$ be the accumulation function. Let $X$ be the amount that must be invested at time $t=0$ to accumulate to 1 dollar at $t=T$. Then
\[
X\cdot a(T)=1
\]
or equivalently, $X= \frac{1}{a(T)}$.\\
The amount $X=\frac{1}{a(T)}$ is the \textbf{present value} of 1 paid at time $T$.
\end{definition}
It follows that the present value of a single payment of $C$ at time $t+T$ is $\frac{C}{a(T)}$.\\
More generally, for a cash flow $\cf{C} = \{(c_1,t_1),(c_2,t_2),\ldots,(c_n,t_n)\}$ consisting of a series of payments, with $c_i$ received at time $t_i$, for $i = 1,2 ,3,\ldots, n$, where $t_1\geq 0$ and $t_i<t_j$ for $i<j$, the present value of this cash flow, denoted by $\PV(\cf{C})$, is defined by
\[
\PV(\cf{C})=\sum_{i=1}^n\frac{c_i}{a(t_i)}
\] 
\begin{definition}[Time Value]
\hfill\\\normalfont The \textbf{time value} of the cash flow $\cf{C}$ at time $t\geq 0$, denoted by $\TV(\cf{C},t)$, is given by
\[
\TV(\cf{C},t) = \PV(\cf{C})\times a(t)
\]
\end{definition}
A consequence of the above definition is that for $0<s<t$,
\[
\TV(\cf{C},t)=\frac{a(t)}{a(s)}\times\TV(\cf{C},s)
\]
\begin{definition}[Principle of Equivalence]
\hfill\\\normalfont In an environment where both the \textit{interest rate} and its \textit{method of accumulation} remain the same over any time period, two cash flows streams are \textbf{equivalent} if and only if they have the same present value.\\(Alternatively, if and only if they have the same time value at $t=T$ for any $T\geq 0$).
\end{definition}
It follows that the cash flow $\cf{C}=\{(c_1,t_1),(c_2,t_2),\ldots,(c_n,t_n)\}$ is equivalent to a single payment of $\PV(\cf{C})=\sum_{i=1}^n\frac{c_i}{a(t_i)}$ at time $t=0$.
\begin{definition}[Deferred Cash Flow]
\hfill\\\normalfont Let $k>0$ and define the cash flow $\cf{C}_{(k)}=\{(c_1,t_1+k),(c_2,t_2+k),\ldots,(c_n,t_n+k)\}$ which is essentially the cash flow $\cf{C} = \{(c_1,t_1),(c_2,t_2),\ldots,(c_n,t_n)\}$ deferred by $k$ years.\\
If the accumulation function is $a(t)$, then
\[
\frac{\PV(\cf{C})}{\PV(\cf{C}_{(k)})}=a(k)
\]
\end{definition}
\textbf{Notations}:\\
For the special case when $t_i = i-1$,
\[
\cf{C}=\{(c_1,0),(c_2,1),\ldots,(c_n,n-1)\}
\]
can be written as $(c_1,c_2,\ldots, c_n)$.
\begin{definition}[Equation of Value]
\hfill\\\normalfont Consider the cash flow stream $\cf{C}=\{(c_1,t_1),)c_2,t_2),\ldots,(c_n,t_n)\}$. The equation
\[
\PV(\cf{C})=\sum_{i=1}^n\frac{c_i}{(1+r)^{t_i}} = 0
\]
is known as the \textbf{equation of value}.
\end{definition}
\begin{definition}[Internal Rate of Return(IRR)]
\hfill\\\normalfont Any non-negative root, $r$ of the equation of value is called the \textbf{yield} or \textbf{internal rate of return (IRR)}, of the cash flow stream.
\end{definition}
\subsection{Annuities}
\begin{definition}[Annuities Immediate and Annuities Due]
\hfill\\\normalfont An annuity is a series of payment made at regular intervals.\\
An \textbf{annuity-due} is one for which payments are made at the \textit{beginning} of each period.\\
An \textbf{annuity-immediate} is one for which payments are made at the \textit{beginning} of each period.
\end{definition}
\begin{definition}[Perpetuity]
\hfill\\\normalfont A \textbf{perpetuity} is an annuity with an infinite number of payments.
\end{definition}
\begin{definition}[Loans]
\hfill\\\normalfont \textbf{Loans} are normally repaid by a series of installment payments made at \textit{periodic} intervals. The size of each installment can be determined using present-value analysis.\\
Specifically, if we let $L$ be the amount of loan taken at time $t=0$ and let $\cf{C} =\{(c_1,t_1),(c_2,t_2),\ldots,(c_n,t_n)\}$ be the series of repayments, then
\[
L :=\PV(\cf{C})
\]
\end{definition}
We can also compute the balance of the loan at any point in time.\\
\begin{definition}[Loan Balance]
\hfill\\\normalfont
The \textbf{loan balance} $L_m^{\text{Balance}}$ immediately after the $m$th installment has been paid is the \textbf{time value} at $t = m$ of the remaining $(n-m)$ installment payments.\\
Suppose installment is paid annually with effectively annual rate $r$ and each repayment of value $c_i$ for year $m+i$, the loan balance
\[
L_m^\text{Balance} = \sum_{i = 1}^{n-m} \frac{c_i}{(1+r)^i}
\]
\end{definition}
Suppose each annual repayment is of value $A$. In reality, the loan is usually fully paid with $n$ repayment of $A$ plus a final payment $B$ made at time $t\geq n$, where $B$ is determined from the equation
\[
L=\PV(0,\underbrace{A,A,\ldots,A}_{n \text{payments}})+\PV(\{(B,t)\})
\]
\section{Bonds and Term Structure}
\subsection{Bond Terminology}
\begin{definition}[Bond]
\hfill\\\normalfont
A \textbf{bond} is a written contract between the issuers(borrowers) and the investers(lenders) which specifies the following:
\begin{itemize}
  \item \textbf{Face value}, $F$, of the bond: the amount based on which periodic interest payments are computed
  \item \textbf{Redemption/maturity value}, $R$, of the bond: the amount to be repaid at the end of the loan
  \item \textbf{Maturity date} of the bond: the date on which the loan will be fully repaid
  \item \textbf{Coupon rate}, $c$, (for coupon-paying bonds): the bond's interest payments, as a percentage of the par value, to be made to investors at regular intervals during the term of the loan
\end{itemize}
\end{definition}
\subsection{Bond Valuations}
We use the following notations in connection with the bond pricing formula that follows.
\begin{itemize}
  \item $P$ = the current price of a bond
  \item $F$ = face value of the bond
  \item $R$ = redemption/maturity value of bond
  \item $c$ = nominal coupon rate
  \item $m$ = number of coupon payments per year
  \item $n$ = total number of coupon payments (number of years $\times m$)
  \item $\lambda$ = nomial yield
\end{itemize}
\begin{theorem}[Price of a Bond]
\hfill\\\normalfont
The price of a bond equals to the present value of the cash flow consisting of all coupon payments and the redemption value at maturity, calculated at yield $\lambda$.\\
For the case when the cash flow is made up of:
\begin{itemize}
  \item coupon payments of $\frac{cF}{m}$ at time $t=\frac{1}{m},\frac{2}{m},\ldots, \frac{n}{m}$ ( a total of $n$ payments)
  \item redemption value $R$ at $t = \frac{n}{m}$
\end{itemize}
We have
\[
P=\frac{R}{\left(1+\frac{\lambda}{m}\right)^n}+\sum_{i=1}^n\frac{\frac{cF}{m}}{\left(1+\frac{\lambda}{m}\right)^i}
\]
\end{theorem}
When $F=R$,
\[
P = F+F\left(\frac{c-\lambda}{\lambda}\right)\left[1-\frac{1}{\left(1+\frac{\lambda}{m}\right)^n}\right]
\]
A bond is said to be priced
\begin{itemize}
  \item{\makebox[4cm]{at a \textbf{premium}\hfill} if $P>F$}
  \item{\makebox[4cm]{at \textbf{par}\hfill} if $P=F$}
  \item{\makebox[4cm]{at a \textbf{discount}\hfill} if $P<F$}
\end{itemize}
From the proceding bond pricing formula, it is clear
\begin{itemize}
  \item \makebox[3cm]{$P>F$\hfill}\makebox[3cm]{if and only if\hfill}\makebox[3cm]{$c>\lambda$}
  \item \makebox[3cm]{$P=F$\hfill}\makebox[3cm]{if and only if\hfill}\makebox[3cm]{$c=\lambda$}
  \item \makebox[3cm]{$P<F$\hfill}\makebox[3cm]{if and only if\hfill}\makebox[3cm]{$c<\lambda$}
\end{itemize}
\begin{theorem}[Makeham Formula]
\hfill\\\normalfont Let $K=\frac{F}{\left(1+\frac{\lambda}{m}\right)^n}$, we have
\[
P=K+\frac{c}{\lambda}(F-K)
\]
\end{theorem}
\begin{theorem}
\hfill\\\normalfont
Let $P_k$ be the price immediately after the $k$ the coupon payment. Then
\[
P_{k+1}=P_{k}\left(1+\frac{\lambda}{m}\right)-\frac{cF}{m}
\]
\end{theorem}
\begin{definition}[Zero Coupon Bonds]
\hfill\\\normalfont \textbf{Zero coupon bonds} are bonds that pay no coupons. The cash flow for a $N$-year zero-coupon bond is the maturity value, $R$ at $t=N$. Hence, at an annual yield of $\lambda$,
\[
P=\frac{R}{(1+\lambda)^N}
\]
\end{definition}
\begin{definition}[Perpetual Bonds]
\hfill\\\normalfont A bond that never matures (i.e., $n\to \infty$) is called a \textbf{perpetual bond}. Clearly,
\[
P=\frac{cF}{\lambda}
\]
\end{definition}
\begin{definition}[Bond Price Between Coupon Payments]
\hfill\\\normalfont The price of a bond traded in $t=\frac{k+\varepsilon}{m}, (0\leq \varepsilon < 1$, which is between $k$th and $k+1$th coupon payment dates is
\[
P_{k+\varepsilon} = (1+\mu)^\varepsilon P_k
\]
where $\mu$ is the effective annual yield of the bond over the period $[k,k+1)$.
\end{definition}
\subsection{Macaulay Duration and Modified Duration}
\begin{definition}[Macaulay Duration]
\hfill\\\normalfont The \textbf{Macaulay duration} is one of the commonly used measures of bond's price sensitivity to changes in interest rate.\\
For cash flow stream $\cf{C}=\{(c_i,t_i)\mid i = 1, 2, \ldots, n\}$, the Macaulay duration, $D$. is defined by
\[
D=\frac{\sum_{i=1}^nt_i\cdot\PV(c_i)}{\sum_{i=1}^n\PV(c_i)}
\]
Equivalently, the Macaulay duration can be defined by the weighted average time to maturity of the cash flow stream:
\[
D=\sum_{i=1}^nw_it_i
\]
where weight $w_i = \frac{\PV(c_i)}{\sum_{j=1}^n\PV(c_j)}$.
\end{definition}
\begin{theorem}[Properties of Macaulay Duration]\hfill\\\normalfont
\begin{itemize}
  \item If $c_i\geq 0$ for all $i$, then $t_0\leq D\leq t_n$.
  \item For a zero-coupon bond, $D=t_n$.
\end{itemize}
\end{theorem}
We can extend definition of Macaulay duration $D$ to any infinite cash flow stream $\cf{C}=\{(c_i,t_i)\mid i = 1, 2,\ldots\}$
\[
D=\frac{\sum_{i=1}^\infty t_i\cdot\PV(c_i)}{\sum_{i=1}^\infty\PV(c_i)}
\] 
\begin{theorem}[Macaulay Duration of bonds]\hfill\\\normalfont
For a bond that pays a total of $n$ coupons at a frequency of $m$ payments a year. Let the nominal bond yield be $\lambda$ and nominal coupon rate be $c$ respectively. The cash flow stream in this case is
\[
\cf{C}=\{(\frac{cF}{m}, t_1),\ldots, (\frac{cF}{m},t_{n-1}),(\frac{cF}{m}+F,t_n)\}
\]
as $t_i = \frac{i}{m}$, so that
\[
D=\frac{1}{P}\left[\sum_{i=1}^n\frac{i}{m}\frac{\frac{cF}{m}}{\left(1+\frac{\lambda}{m}\right)^i}+\frac{n}{m}\frac{F}{\left(1+\frac{\lambda}{m}\right)^n}\right]
\]
where
\[
P=\sum_{i=1}^n\frac{\frac{cF}{m}}{\left(1+\frac{\lambda}{m}\right)^i}+\frac{F}{\left(1+\frac{\lambda}{m}\right)^n}
\]
Let $\mu = \frac{\lambda}{m}$ and $\gamma = \frac{c}{m}$, then
\[
D = \frac{\sum_{i=1}^n\frac{i}{m}\frac{\gamma}{(1+\mu)^i}+\frac{n}{m}\frac{1}{(1+\mu)^n}}{\sum_{i=1}^n\frac{\gamma}{(1+\mu)^i}+\frac{1}{(1+\mu)^n}}
\]
It can be shown that
\[
D = \frac{1+\mu}{m\mu}-\frac{1+\mu+n(\gamma-\mu)}{m\mu+m\gamma\left[\left(1+\mu)^n-1\right)\right]}
\]
\end{theorem}
As the time to maturity tends to infinity, i.e. $n\to\infty$, for a perpetual bond,
\[
D=\frac{1+\mu}{m\mu}
\]
Macalay duration measures the sensitivity of bond prices to interest rates.\\
To see this, differentiate the pricing formula, we will have
\[
\frac{\diff P}{\diff \lambda}=\left(-\frac{1}{1+\frac{\lambda}{m}}D\right)P
\]
\begin{definition}[Modified duration]
\hfill\\\normalfont The term $\frac{1}{1+\frac{\lambda}{m}}D$ is defined as the \textbf{modified duration} and is denoted by $D_\text{M}$.
\end{definition}
In general, for a cash flow $\cf{C}=\{(c_i, t_i)\mid i = 1, 2,\ldots, n\}$ at an effective annual rate of $r$, the relation 
\[
\frac{\diff P}{\diff r} = -D_\text{M}
\]
still holds.
\begin{theorem}[Linear Approximation of Price Change]
\hfill\\\normalfont If $\Delta\lambda$ is a small change in $\lambda$, then
\[
\Delta P = -D_\text{M}P\Delta\lambda
\]
\end{theorem}
\begin{definition}[Duration of Bond Portfolio]
\hfill\\\normalfont Consider a bond portfolio consisting of $\alpha_i$ units of bond $i$, $i = 1, 2, \ldots, n$, assuming that the bonds have a \textit{common} effective annual yield to maturity.\\
Let $P_i$ and $D_i$ be respectively the price and duration of bond $i$. Then, the \textbf{duration} $D_p$ \textbf{of a portfolio} of $n$ bonds of equal yield to maturity, $\lambda$ is given by
\[
D_p = \sum_{i=1}^nw_iD_i
\]
where the \textbf{portfolio weight} $w_i = \frac{\alpha_i P_i}{\sum_{i=1}^n\alpha_i P_i}$
\end{definition}
\begin{definition}[Convexity $C$]
\hfill\\\normalfont \textbf{Convexity} of the bond $C$, is defined as the second derivative of the bond price with respect to bond yield, divided by the price of the bond.
\[
C:=\frac{\frac{\diff^2 P}{\diff \lambda^2}}{P}
\]
\end{definition}
By Taylor series, it can be show that
\[
\Delta P \approx -D_MP\Delta\lambda+\frac{1}{2}\frac{\diff^2 P}{\diff \lambda^2}(\Delta \lambda)^2
\]
Therefore,
\[
\Delta P \approx P\left[-D_M\Delta\lambda + \frac{1}{2}C(\Delta\lambda)^2\right]
\]
This obtains a better approximation of the change in price.\\
Also, from the bond pricing formula $P=\sum_{i=1}^n\frac{c_i}{\left(1+\frac{\lambda}{m}\right)^i}$, we have
\begin{align*}
C&=\frac{\frac{\diff^2 P}{\diff \lambda^2}}{P} \\
&=\frac{1}{Pm^2\left(1+\frac{\lambda}{m}\right)^2}\sum_{i=1}^n i(i+1)\frac{c_i}{\left(1+\frac{\lambda}{m}\right)^i}\\
&=\frac{F}{P}\{\frac{2c}{\lambda^3}\left(1-\frac{1}{\left(1+\frac{\lambda}{m}\right)^n}\right)\\&-\frac{2nc}{m\lambda^2\left(1+\frac{\lambda}{m}\right)^{n+1}}-\frac{n(n+1)(c-\lambda)}{\lambda m^2\left(1+\frac{\lambda}{m}\right)^{n+2}}\}
\end{align*}
\subsection{Yield curves and Term Structure of Interest Rates}
\begin{definition}[Spot Rates]
\hfill\\\normalfont A \textbf{spot rate} is the \textit{annual} interest rate that begins today ($t=0$) and lasts until some future time $t$. We denote this rate by $s_t$.\\In effect the spot rate $s_t$ is the yield to maturity of a zero-coupon bond that matures at $t$.
\end{definition}
\begin{definition}[Forward Rate]
\hfill\\\normalfont The interest rate observed at some future time $t_1>0$ and lasts until a time $t_2>t_1$ is called a \textbf{forward rate}, denoted by $f_{t_1,t_2}$. \\Note that $f_{0,t}=s_t$
\end{definition}
\begin{theorem}
\hfill\\\normalfont In general,
\[
(1+s_k)^k = (1+s_j)^j(1+f_{j,k})^{k-j}
\] 
and
\[
(1+s_n)^n = (1+s_1)(1+f_{1,2})(1+f_{2,3})\cdots(1+f_{n-1,n})
\]
\end{theorem}


\end{document}