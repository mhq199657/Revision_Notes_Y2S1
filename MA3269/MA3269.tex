\PassOptionsToPackage{svgnames}{xcolor}
\documentclass[12pt]{article}



\usepackage[margin=1in]{geometry}  
\usepackage{graphicx}             
\usepackage{amsmath}              
\usepackage{amsfonts}              
\usepackage{framed}               
\usepackage{amssymb}
\usepackage{array}
\usepackage{amsthm}
\usepackage[nottoc]{tocbibind}
\usepackage{bm}
\usepackage{enumitem}


  \newcommand\norm[1]{\left\lVert#1\right\rVert}
\setlength{\parindent}{0cm}
\setlength{\parskip}{0em}
\newcommand{\Lim}[1]{\raisebox{0.5ex}{\scalebox{0.8}{$\displaystyle \lim_{#1}\;$}}}
\newtheorem{definition}{Definition}[section]
\newtheorem{theorem}{Theorem}[section]
\newtheorem{notation}{Notation}[section]
\theoremstyle{definition}
\DeclareMathOperator{\arcsec}{arcsec}
\DeclareMathOperator{\arccot}{arccot}
\DeclareMathOperator{\arccsc}{arccsc}
\DeclareMathOperator{\PV}{PV}
\DeclareMathOperator{\TV}{TV}
\DeclareMathOperator{\diff}{d}
\DeclareMathOperator{\expec}{E}
\DeclareMathOperator{\var}{Var}
\DeclareMathOperator{\cov}{Cov}
\DeclareMathOperator{\CE}{CE}
\DeclareMathOperator{\RP}{RP}
\newcommand\cf[1]{\mathbf{#1}}
\setcounter{tocdepth}{1}
\setcounter{section}{-1}
\begin{document}

\title{Revision notes - MA3269}
\author{Ma Hongqiang}
\maketitle
\tableofcontents

\clearpage
%\twocolumn
\section{Preliminary Result}
\subsection{Summation of series}
\begin{itemize}
  \item $\sum_{i=0}^n y^i = \frac{1-y^{n+1}}{1-y}$
  \item $\sum_{i=0}^\infty y^i = \frac{1}{1-y}\text{ provided }|r|<1$.
  \item $\sum_{i=1}^n iy^{i-1} = \frac{1-y^n(1+n-ny)}{(1-y)^2}$
  \item $\sum_{i=1}^\infty iy^{i-1} = \frac{1}{(1-y)^2}\text{ provided }|r|<1$.
  \item $\sum_{i=1}^n iy^{i} = \frac{y(1-y^n)-ny^{n+1}(1-y)}{(1-y)^2}$
  \item $\sum_{i=1}^\infty iy^{i} = \frac{y}{(1-y)^2}\text{ provided }|r|<1$.  
  \item $\sum_{i=1}^n i = \frac{1}{2}n(n+1)$
  \item $\sum_{i=1}^n i^2 = \frac{1}{6}n(n+1)(2n+1)$
  \item $\sum_{i=1}^n i^3 = \frac{1}{4}n^2(n+1)^2$
\end{itemize}
\subsection{Newton Rhapson Method}
\[
x_{i+1} = x_i-\frac{f(x_i)}{f^\prime(x_i)}
\]
\subsection{Force of Interest}
Accumulation function $a(s,t)$ can be derived from \textbf{force of interest} as such:
\[
a(s,t) = e^{\int_s^t \delta(r)\diff r}
\]
where $\delta(r)$ is the force of interest with $s\leq r\leq t$.
\subsection{Standard Derivatives}

\clearpage
\section{Theory of Interest}
\subsection{Interest}
\begin{definition}[Accumulation Function]
\hfill\\\normalfont When a principal of 1 dollar is deposited in an interest-paying account at time $t=0$, it earns some interest over the time interval $[0,t]$. \\
The accumulated value of 1 dollar at time $t \geq 0$, denoted by $a(t)$, is known as the \textbf{accumulation function}. Clearly, $a(0)=1$.
\end{definition}
\begin{definition}[Simple and Compound Interest]
\hfill\\\normalfont Let $r$ be the annual rate of interest.\\
Based on the \textbf{simple-interest} method of calculating interest,
\[
a(t)=1+rt\;\;\;\text{for }t \geq 0
\]
If the \textbf{compound interest} method is used,
\[
a(t)=(1+r)^t \;\;\;\text{for }t\geq 0
\]
\end{definition}
Suppose the interest rate is $r_i$ for the period $[\sum_{k=0}^{i-1} t_i,\sum_{k=1}^{i} t_{i} ]$, where $t_0 = 0$, 
\[
a(t_j)=1+\sum_{i=1}^j r_it_i \;\;\;\text{when simple interest is used;}
\]
\[
a(t_j)=\prod_{i=1}^j (1+r_i)^{t_i}\;\;\;\text{when compound interest is used;}
\]
\begin{definition}[Frequency of Compounding]
\hfill\\\normalfont When an interest of $r = r^{(p)}$ is paid $p$ times a year (or equivalently, $r^{(p)}$ is \textbf{convertible} $p$\textbf{thly} or $r^{(p)}$ is compounded $p$ times a year), we call $p$ the \textbf{frequency of compounding} and $r^{(p)}$ the \textbf{nominal} rate of interest.
\end{definition}
The interest to be paid over the period, is $\frac{r^{(p)}}{p}$. Effectively, \$1 invested at time $t=0$ will grow to $\left(1+\frac{r^{(p)}}{p}\right)$ over a period of length $\frac{1}{p}$, so that the accumulated amount after one year is$\left(1+\frac{r^{(p)}}{p}\right)^p$.\\
\textbf{Remarks}
\begin{enumerate}
  \item We write the superscript $(p)$ for $r^{(p)}$ to indicate the frequency of compounding $p$.
  \item We can drop the superscript $(p)$ when $p=1$.
  \item $p = 2, 4, 12 $ correspond to semi-annual, quarterly and monthly compounding respectively,
\end{enumerate}
\begin{definition}[Equivalent Interest Rates]
\hfill\\\normalfont Two nominal interest rates are said to be \textbf{equivalent} if and only if they yield same accumulation amount over a year. Hence, the nominal rates $r^{(p)}$ and $r^{(q)}$ are equivalent if and only if
\[
\left(1+\frac{r^{(p)}}{p}\right)^p=\left(1+\frac{r^{(q)}}{q}\right)^q
\]
In particular, the \textbf{effective} annual interest rate (when $p=1$), denoted by $r_e$, is given by
\[
1+r_e = \left(1+\frac{r^{(p)}}{p}\right)^p
\]
The corresponding accumulation function is
\[
a(t) = (1+r_e)^t = \left(1+\frac{r^{(p)}}{p}\right)^{pt}
\]
\end{definition}
It can be shown that $r_e\geq r^{(p)}$ for $p>1$.
\begin{definition}[Continuous Compounding]
\hfill\\\normalfont The interest is \textbf{compounded continuously} when the frequency of compounding tends to infinity.\\
Let $r^{(\infty)}$ denote the nominal rate of interest under continuous compounding. Then,
\[
a(1)=\Lim{p\to\infty}\left(1+\frac{r^{(\infty)}}{p}\right)^p = e^{r^{(\infty)}}
\]
\end{definition}
The number $r^{(\infty)}$ is known as the \textbf{continuously compounded} rate of interest.
The corresponding accumulatio function is
\[
a(t) = e^{r^{(\infty)}t},\;\;\;t\geq 0
\]
Note that $e^{r^{(\infty)}}= 1+r_e$.\\
It can be shown that
\[
e^r>\left(1+\frac{r}{p}\right)^p
\]
for any $r>0$ and for any $p\in\mathbb{Z}^+$.
\subsection{Present Value}
\begin{definition}[Present Value, Time Value]
\hfill\\\normalfont Let $a(t)$ be the accumulation function. Let $X$ be the amount that must be invested at time $t=0$ to accumulate to 1 dollar at $t=T$. Then
\[
X\cdot a(T)=1
\]
or equivalently, $X= \frac{1}{a(T)}$.\\
The amount $X=\frac{1}{a(T)}$ is the \textbf{present value} of 1 paid at time $T$.
\end{definition}
It follows that the present value of a single payment of $C$ at time $t+T$ is $\frac{C}{a(T)}$.\\
More generally, for a cash flow $\cf{C} = \{(c_1,t_1),(c_2,t_2),\ldots,(c_n,t_n)\}$ consisting of a series of payments, with $c_i$ received at time $t_i$, for $i = 1,2 ,3,\ldots, n$, where $t_1\geq 0$ and $t_i<t_j$ for $i<j$, the present value of this cash flow, denoted by $\PV(\cf{C})$, is defined by
\[
\PV(\cf{C})=\sum_{i=1}^n\frac{c_i}{a(t_i)}
\] 
\begin{definition}[Time Value]
\hfill\\\normalfont The \textbf{time value} of the cash flow $\cf{C}$ at time $t\geq 0$, denoted by $\TV(\cf{C},t)$, is given by
\[
\TV(\cf{C},t) = \PV(\cf{C})\times a(t)
\]
\end{definition}
A consequence of the above definition is that for $0<s<t$,
\[
\TV(\cf{C},t)=\frac{a(t)}{a(s)}\times\TV(\cf{C},s)
\]
\begin{definition}[Principle of Equivalence]
\hfill\\\normalfont In an environment where both the \textit{interest rate} and its \textit{method of accumulation} remain the same over any time period, two cash flows streams are \textbf{equivalent} if and only if they have the same present value.\\(Alternatively, if and only if they have the same time value at $t=T$ for any $T\geq 0$).
\end{definition}
It follows that the cash flow $\cf{C}=\{(c_1,t_1),(c_2,t_2),\ldots,(c_n,t_n)\}$ is equivalent to a single payment of $\PV(\cf{C})=\sum_{i=1}^n\frac{c_i}{a(t_i)}$ at time $t=0$.
\begin{definition}[Deferred Cash Flow]
\hfill\\\normalfont Let $k>0$ and define the cash flow $\cf{C}_{(k)}=\{(c_1,t_1+k),(c_2,t_2+k),\ldots,(c_n,t_n+k)\}$ which is essentially the cash flow $\cf{C} = \{(c_1,t_1),(c_2,t_2),\ldots,(c_n,t_n)\}$ deferred by $k$ years.\\
If the accumulation function is $a(t)$, then
\[
\frac{\PV(\cf{C})}{\PV(\cf{C}_{(k)})}=a(k)
\]
\end{definition}
\textbf{Notations}:\\
For the special case when $t_i = i-1$,
\[
\cf{C}=\{(c_1,0),(c_2,1),\ldots,(c_n,n-1)\}
\]
can be written as $(c_1,c_2,\ldots, c_n)$.
\begin{definition}[Equation of Value]
\hfill\\\normalfont Consider the cash flow stream $\cf{C}=\{(c_1,t_1),)c_2,t_2),\ldots,(c_n,t_n)\}$. The equation
\[
\PV(\cf{C})=\sum_{i=1}^n\frac{c_i}{(1+r)^{t_i}} = 0
\]
is known as the \textbf{equation of value}.
\end{definition}
\begin{definition}[Internal Rate of Return(IRR)]
\hfill\\\normalfont Any non-negative root, $r$ of the equation of value is called the \textbf{yield} or \textbf{internal rate of return (IRR)}, of the cash flow stream.
\end{definition}
\subsection{Annuities}
\begin{definition}[Annuities Immediate and Annuities Due]
\hfill\\\normalfont An annuity is a series of payment made at regular intervals.\\
An \textbf{annuity-due} is one for which payments are made at the \textit{beginning} of each period.\\
An \textbf{annuity-immediate} is one for which payments are made at the \textit{beginning} of each period.
\end{definition}
\begin{definition}[Perpetuity]
\hfill\\\normalfont A \textbf{perpetuity} is an annuity with an infinite number of payments.
\end{definition}
\begin{definition}[Loans]
\hfill\\\normalfont \textbf{Loans} are normally repaid by a series of installment payments made at \textit{periodic} intervals. The size of each installment can be determined using present-value analysis.\\
Specifically, if we let $L$ be the amount of loan taken at time $t=0$ and let $\cf{C} =\{(c_1,t_1),(c_2,t_2),\ldots,(c_n,t_n)\}$ be the series of repayments, then
\[
L :=\PV(\cf{C})
\]
\end{definition}
We can also compute the balance of the loan at any point in time.\\
\begin{definition}[Loan Balance]
\hfill\\\normalfont
The \textbf{loan balance} $L_m^{\text{Balance}}$ immediately after the $m$th installment has been paid is the \textbf{time value} at $t = m$ of the remaining $(n-m)$ installment payments.\\
Suppose installment is paid annually with effectively annual rate $r$ and each repayment of value $c_i$ for year $m+i$, the loan balance
\[
L_m^\text{Balance} = \sum_{i = 1}^{n-m} \frac{c_i}{(1+r)^i}
\]
\end{definition}
Suppose each annual repayment is of value $A$. In reality, the loan is usually fully paid with $n$ repayment of $A$ plus a final payment $B$ made at time $t\geq n$, where $B$ is determined from the equation
\[
L=\PV(0,\underbrace{A,A,\ldots,A}_{n \text{payments}})+\PV(\{(B,t)\})
\]
\clearpage
\section{Bonds and Term Structure}
\subsection{Bond Terminology}
\begin{definition}[Bond]
\hfill\\\normalfont
A \textbf{bond} is a written contract between the issuers(borrowers) and the investers(lenders) which specifies the following:
\begin{itemize}
  \item \textbf{Face value}, $F$, of the bond: the amount based on which periodic interest payments are computed
  \item \textbf{Redemption/maturity value}, $R$, of the bond: the amount to be repaid at the end of the loan
  \item \textbf{Maturity date} of the bond: the date on which the loan will be fully repaid
  \item \textbf{Coupon rate}, $c$, (for coupon-paying bonds): the bond's interest payments, as a percentage of the par value, to be made to investors at regular intervals during the term of the loan
\end{itemize}
\end{definition}
\subsection{Bond Valuations}
We use the following notations in connection with the bond pricing formula that follows.
\begin{itemize}
  \item $P$ = the current price of a bond
  \item $F$ = face value of the bond
  \item $R$ = redemption/maturity value of bond
  \item $c$ = nominal coupon rate
  \item $m$ = number of coupon payments per year
  \item $n$ = total number of coupon payments (number of years $\times m$)
  \item $\lambda$ = nomial yield
\end{itemize}
\begin{theorem}[Price of a Bond]
\hfill\\\normalfont
The price of a bond equals to the present value of the cash flow consisting of all coupon payments and the redemption value at maturity, calculated at yield $\lambda$.\\
For the case when the cash flow is made up of:
\begin{itemize}
  \item coupon payments of $\frac{cF}{m}$ at time $t=\frac{1}{m},\frac{2}{m},\ldots, \frac{n}{m}$ ( a total of $n$ payments)
  \item redemption value $R$ at $t = \frac{n}{m}$
\end{itemize}
We have
\[
P=\frac{R}{\left(1+\frac{\lambda}{m}\right)^n}+\sum_{i=1}^n\frac{\frac{cF}{m}}{\left(1+\frac{\lambda}{m}\right)^i}
\]
\end{theorem}
When $F=R$,
\[
P = F+F\left(\frac{c-\lambda}{\lambda}\right)\left[1-\frac{1}{\left(1+\frac{\lambda}{m}\right)^n}\right]
\]
A bond is said to be priced
\begin{itemize}
  \item{\makebox[4cm]{at a \textbf{premium}\hfill} if $P>F$}
  \item{\makebox[4cm]{at \textbf{par}\hfill} if $P=F$}
  \item{\makebox[4cm]{at a \textbf{discount}\hfill} if $P<F$}
\end{itemize}
From the proceding bond pricing formula, it is clear
\begin{itemize}
  \item \makebox[3cm]{$P>F$\hfill}\makebox[3cm]{if and only if\hfill}\makebox[3cm]{$c>\lambda$}
  \item \makebox[3cm]{$P=F$\hfill}\makebox[3cm]{if and only if\hfill}\makebox[3cm]{$c=\lambda$}
  \item \makebox[3cm]{$P<F$\hfill}\makebox[3cm]{if and only if\hfill}\makebox[3cm]{$c<\lambda$}
\end{itemize}
\begin{theorem}[Makeham Formula]
\hfill\\\normalfont Let $K=\frac{F}{\left(1+\frac{\lambda}{m}\right)^n}$, we have
\[
P=K+\frac{c}{\lambda}(F-K)
\]
\end{theorem}
\begin{theorem}
\hfill\\\normalfont
Let $P_k$ be the price immediately after the $k$ the coupon payment. Then
\[
P_{k+1}=P_{k}\left(1+\frac{\lambda}{m}\right)-\frac{cF}{m}
\]
\end{theorem}
\begin{definition}[Zero Coupon Bonds]
\hfill\\\normalfont \textbf{Zero coupon bonds} are bonds that pay no coupons. The cash flow for a $N$-year zero-coupon bond is the maturity value, $R$ at $t=N$. Hence, at an annual yield of $\lambda$,
\[
P=\frac{R}{(1+\lambda)^N}
\]
\end{definition}
\begin{definition}[Perpetual Bonds]
\hfill\\\normalfont A bond that never matures (i.e., $n\to \infty$) is called a \textbf{perpetual bond}. Clearly,
\[
P=\frac{cF}{\lambda}
\]
\end{definition}
\begin{definition}[Bond Price Between Coupon Payments]
\hfill\\\normalfont The price of a bond traded in $t=\frac{k+\varepsilon}{m}, (0\leq \varepsilon < 1$, which is between $k$th and $k+1$th coupon payment dates is
\[
P_{k+\varepsilon} = (1+\mu)^\varepsilon P_k
\]
where $\mu$ is the effective annual yield of the bond over the period $[k,k+1)$.
\end{definition}
\subsection{Macaulay Duration and Modified Duration}
\begin{definition}[Macaulay Duration]
\hfill\\\normalfont The \textbf{Macaulay duration} is one of the commonly used measures of bond's price sensitivity to changes in interest rate.\\
For cash flow stream $\cf{C}=\{(c_i,t_i)\mid i = 1, 2, \ldots, n\}$, the Macaulay duration, $D$. is defined by
\[
D=\frac{\sum_{i=1}^nt_i\cdot\PV(c_i)}{\sum_{i=1}^n\PV(c_i)}
\]
Equivalently, the Macaulay duration can be defined by the weighted average time to maturity of the cash flow stream:
\[
D=\sum_{i=1}^nw_it_i
\]
where weight $w_i = \frac{\PV(c_i)}{\sum_{j=1}^n\PV(c_j)}$.
\end{definition}
\begin{theorem}[Properties of Macaulay Duration]\hfill\\\normalfont
\begin{itemize}
  \item If $c_i\geq 0$ for all $i$, then $t_0\leq D\leq t_n$.
  \item For a zero-coupon bond, $D=t_n$.
\end{itemize}
\end{theorem}
We can extend definition of Macaulay duration $D$ to any infinite cash flow stream $\cf{C}=\{(c_i,t_i)\mid i = 1, 2,\ldots\}$
\[
D=\frac{\sum_{i=1}^\infty t_i\cdot\PV(c_i)}{\sum_{i=1}^\infty\PV(c_i)}
\] 
\begin{theorem}[Macaulay Duration of bonds]\hfill\\\normalfont
For a bond that pays a total of $n$ coupons at a frequency of $m$ payments a year. Let the nominal bond yield be $\lambda$ and nominal coupon rate be $c$ respectively. The cash flow stream in this case is
\[
\cf{C}=\{(\frac{cF}{m}, t_1),\ldots, (\frac{cF}{m},t_{n-1}),(\frac{cF}{m}+F,t_n)\}
\]
as $t_i = \frac{i}{m}$, so that
\[
D=\frac{1}{P}\left[\sum_{i=1}^n\frac{i}{m}\frac{\frac{cF}{m}}{\left(1+\frac{\lambda}{m}\right)^i}+\frac{n}{m}\frac{F}{\left(1+\frac{\lambda}{m}\right)^n}\right]
\]
where
\[
P=\sum_{i=1}^n\frac{\frac{cF}{m}}{\left(1+\frac{\lambda}{m}\right)^i}+\frac{F}{\left(1+\frac{\lambda}{m}\right)^n}
\]
Let $\mu = \frac{\lambda}{m}$ and $\gamma = \frac{c}{m}$, then
\[
D = \frac{\sum_{i=1}^n\frac{i}{m}\frac{\gamma}{(1+\mu)^i}+\frac{n}{m}\frac{1}{(1+\mu)^n}}{\sum_{i=1}^n\frac{\gamma}{(1+\mu)^i}+\frac{1}{(1+\mu)^n}}
\]
It can be shown that
\[
D = \frac{1+\mu}{m\mu}-\frac{1+\mu+n(\gamma-\mu)}{m\mu+m\gamma\left[\left(1+\mu)^n-1\right)\right]}
\]
\end{theorem}
As the time to maturity tends to infinity, i.e. $n\to\infty$, for a perpetual bond,
\[
D=\frac{1+\mu}{m\mu}
\]
Macalay duration measures the sensitivity of bond prices to interest rates.\\
To see this, differentiate the pricing formula, we will have
\[
\frac{\diff P}{\diff \lambda}=\left(-\frac{1}{1+\frac{\lambda}{m}}D\right)P
\]
\begin{definition}[Modified duration]
\hfill\\\normalfont The term $\frac{1}{1+\frac{\lambda}{m}}D$ is defined as the \textbf{modified duration} and is denoted by $D_\text{M}$.
\end{definition}
In general, for a cash flow $\cf{C}=\{(c_i, t_i)\mid i = 1, 2,\ldots, n\}$ at an effective annual rate of $r$, the relation 
\[
\frac{\diff P}{\diff r} = -D_\text{M}
\]
still holds.
\begin{theorem}[Linear Approximation of Price Change]
\hfill\\\normalfont If $\Delta\lambda$ is a small change in $\lambda$, then
\[
\Delta P = -D_\text{M}P\Delta\lambda
\]
\end{theorem}
\begin{definition}[Duration of Bond Portfolio]
\hfill\\\normalfont Consider a bond portfolio consisting of $\alpha_i$ units of bond $i$, $i = 1, 2, \ldots, n$, assuming that the bonds have a \textit{common} effective annual yield to maturity.\\
Let $P_i$ and $D_i$ be respectively the price and duration of bond $i$. Then, the \textbf{duration} $D_p$ \textbf{of a portfolio} of $n$ bonds of equal yield to maturity, $\lambda$ is given by
\[
D_p = \sum_{i=1}^nw_iD_i
\]
where the \textbf{portfolio weight} $w_i = \frac{\alpha_i P_i}{\sum_{i=1}^n\alpha_i P_i}$
\end{definition}
\begin{definition}[Convexity $C$]
\hfill\\\normalfont \textbf{Convexity} of the bond $C$, is defined as the second derivative of the bond price with respect to bond yield, divided by the price of the bond.
\[
C:=\frac{\frac{\diff^2 P}{\diff \lambda^2}}{P}
\]
\end{definition}
By Taylor series, it can be show that
\[
\Delta P \approx -D_MP\Delta\lambda+\frac{1}{2}\frac{\diff^2 P}{\diff \lambda^2}(\Delta \lambda)^2
\]
Therefore,
\[
\Delta P \approx P\left[-D_M\Delta\lambda + \frac{1}{2}C(\Delta\lambda)^2\right]
\]
This obtains a better approximation of the change in price.\\
Also, from the bond pricing formula $P=\sum_{i=1}^n\frac{c_i}{\left(1+\frac{\lambda}{m}\right)^i}$, we have
\begin{align*}
C&=\frac{\frac{\diff^2 P}{\diff \lambda^2}}{P} \\
&=\frac{1}{Pm^2\left(1+\frac{\lambda}{m}\right)^2}\sum_{i=1}^n i(i+1)\frac{c_i}{\left(1+\frac{\lambda}{m}\right)^i}\\
&=\frac{F}{P}\left\{\frac{2c}{\lambda^3}\left(1-\frac{1}{\left(1+\frac{\lambda}{m}\right)^n}\right)-\frac{2nc}{m\lambda^2\left(1+\frac{\lambda}{m}\right)^{n+1}}-\frac{n(n+1)(c-\lambda)}{\lambda m^2\left(1+\frac{\lambda}{m}\right)^{n+2}}\right\}
\end{align*}
\subsection{Yield curves and Term Structure of Interest Rates}
\begin{definition}[Spot Rates]
\hfill\\\normalfont A \textbf{spot rate} is the \textit{annual} interest rate that begins today ($t=0$) and lasts until some future time $t$. We denote this rate by $s_t$.\\In effect the spot rate $s_t$ is the yield to maturity of a zero-coupon bond that matures at $t$.
\end{definition}
\begin{definition}[Forward Rate]
\hfill\\\normalfont The interest rate observed at some future time $t_1>0$ and lasts until a time $t_2>t_1$ is called a \textbf{forward rate}, denoted by $f_{t_1,t_2}$. \\Note that $f_{0,t}=s_t$
\end{definition}
\begin{theorem}
\hfill\\\normalfont In general,
\[
(1+s_k)^k = (1+s_j)^j(1+f_{j,k})^{k-j}
\] 
and
\[
(1+s_n)^n = (1+s_1)(1+f_{1,2})(1+f_{2,3})\cdots(1+f_{n-1,n})
\]
\end{theorem}
\clearpage
\section{Expected Utility Theory}
\subsection{Expected Utility and Risk Attitude}
\begin{definition}[Expected Utility]
\hfill\\\normalfont An individual with an initial wealth of $w_0$ is considering a \textbf{risky prospect}with a random payoff $X$. He is assumed to have a \textbf{utility function} that is real-valued, continuous and \textbf{increasing}. He will make his investment decision based on the \textbf{expected utility} of his final wealth $W:=X+w_0$, defined as follows.
\begin{itemize}
  \item \textbf{Discrete $X$}\\If the risky investment has $n$ possible mutually exclusive payoffs $(x_1,x_2,\ldots, x_n)$ with associated probabilities $p_1,p_2,\ldots, p_n)$, where $\sum_{i=1}^n p_i = 1$, then the \textbf{expected utility} of the individual's final wealth $W$, is given by
  \[
\expec[ U (W)]=\expec[ U (X+w_0)] := \sum_{i=1}^n p_i U (x_i+w_0)
  \]
\item If $X$ is a continuous random variable having a density function $f:(a,b)\to(0,\infty)$, then
\[
\expec[ U (X+w_0)]:=\int_a^b f(x) U (x+w_0)\diff x
\]
\end{itemize}
\end{definition}
\begin{definition}[Utility-based Decision]\hfill\\\normalfont
Under \textbf{utility-based decision}, he individual will
\begin{itemize}
  \item \makebox[8cm][l]{invest in the risky prospect}\makebox[1cm]{if}$\expec[ U (X+w_0)]> U (w_0)$.
  \item \makebox[8cm][l]{avoid the risky prospect}\makebox[1cm]{if}$\expec[ U (X+w_0)]< U (w_0)$.
    \item \makebox[8cm][l]{be indifferent to the risky prospect}\makebox[1cm]{if}$\expec[ U (X+w_0)]= U (w_0)$.
  \end{itemize}
  \end{definition}
Given a set of risky prospects, an individual will \textit{most} favour the one that maximises the expected utility of his final wealth.
\begin{definition}[Characterisation of Risk Attitude]\hfill\\\normalfont
An individual with utility function $U$ is said to be
\begin{itemize}
  \item \makebox[3cm][l]{risk averse}\makebox[1cm]{if}$U$ is strictly concave.\footnote{A function $U$ is strictly concave on $I$ if $U^{\prime\prime}<0$ on $I$.}
  \item \makebox[3cm][l]{risk neutral}\makebox[1cm]{if}$U$ is linear.
  \item \makebox[3cm][l]{risk loving}\makebox[1cm]{if}$U$ is strictly convex.
\end{itemize}
\end{definition}
By Jensen Inequality, we deduce
\begin{theorem}[Equivalent condition for Risk Attitude Characterisation]
\hfill\\\normalfont 
\begin{itemize}
  \item \makebox[3cm][l]{risk averse}\makebox[1cm]{if}$\expec[ U (W)]< U [\expec(W)]$.
  \item \makebox[3cm][l]{risk neutral}\makebox[1cm]{if}$\expec[ U (W)]= U [\expec(W)]$
  \item \makebox[3cm][l]{risk loving}\makebox[1cm]{if}$\expec[ U (W)]> U [\expec(W)]$
\end{itemize}
for \textbf{any} risky investment that yields a final wealth of $W$.
\end{theorem}
\begin{definition}[Positive Affine Transformation]\hfill\\\normalfont
Let $ U $ be an utility function. For any $\alpha>0, \beta\in\mathbb{R}$, the function $\alpha U +\beta$ is a \textbf{positive affine transformation} of $ U $.
\end{definition}
Obviously, both function have the same attitude towards risks.
\subsection{Certainty Equivalent}
\begin{definition}[Certainty Equivalent]
\hfill\\\normalfont Let $U$ be the utility function of an individual. Given a risky prospect with payoff $X$ ,the \textbf{certainty equivalent} of $X$ with respect to $U$, is defined to be the real number $c = ]CE(X;U)$ for which
\[
U(c)=\expec(U(w_0+X))
\]
\end{definition}
It follows that an individual
\begin{itemize}
  \item \makebox[5.5cm][l]{invests in the risky prospect}\makebox[1cm]{if}$\CE(X;U)>w_0$
  \item \makebox[5.5cm][l]{avoids the risky prospect}\makebox[1cm]{if}$\CE(X;U)<w_0$
  \item \makebox[5.5cm][l]{is indifferent}\makebox[1cm]{if}$\CE(X;U)=w_0$
\end{itemize}
For positive affine transformation $\alpha U+\beta$ where $\alpha>0$, we have
\[
\CE(X, \alpha U+\beta) = \CE(X,U)
\]
\begin{definition}[Risk Premium]
\hfill\\\normalfont The \textbf{risk premium} of a risky prospect with respect to an utility function $U$ is the real number $r=\RP(X;U)$ for which
\[
U(w_0-r) = \expec(U(w_0+X))
\]
where $X$ is the payoff.
\end{definition}
Clearly,
\[
r=w_0-c
\]
and hence, an individual
\begin{itemize}
  \item \makebox[5.5cm][l]{invests in the risky prospect}\makebox[1cm]{if}$\RP(X;U)<0$
  \item \makebox[5.5cm][l]{avoids the risky prospect}\makebox[1cm]{if}$\RP(X;U)>0$
  \item \makebox[5.5cm][l]{is indifferent}\makebox[1cm]{if}$\RP(X;U)=0$
\end{itemize}
\subsection{Arrow-Pratt Measures of Risk Aversion}
\begin{definition}[Absolute Risk Aversion]
\hfill\\\normalfont For a \textit{risk averse} individual whose utility function is $U$, his \textbf{Arrow-Pratt absolute risk aversion Coefficient}(ARA) at wealth level $w$ is
\[
-\frac{U^{\prime\prime}(w)}{U^\prime(w)}
\]
\end{definition}
\begin{theorem}[ARA of positive affine transformation]
\hfill\\\normalfont $U_\text{ARA} = V_\text{ARA}$ if and only if $U$ and $V$ are positive affine transformation of each other.
\end{theorem}
We can say that two utility functions are \textbf{equivalent} if and only if they have the same ARA.\\
Suppose two individuals with utility functions $U$ and $V$ admits the following condition:
\[
U_\text{ARA}(w)>V_\text{ARA}(w)
\]
at \textbf{all} wealth level, $w$, we say the individual with utility function $U$ is \textbf{globally more risk averse} than the individual with utility function $V$.
\begin{theorem}
\hfill\\\normalfont More generally, an individual with utility function $U$ is \textbf{globally more risk averse} than an individual with utility function $V$ if and only if there is an increasing and strictly concave function $g$ such that
\[
U(w)=g(V(w))
\]
\end{theorem}
\subsection{Portfolio Selection}
An individual with an initial wealth of $w_0$ can invest a portion (say $\alpha w_0$, where $\alpha\in[0,1]$) of his money in a risky investment $X$ that has a random \textbf{rate of return}, $R$. The expected utility of his final wealth is
\[
\expec[U(W)]=\expec[U(w_0(1+\alpha R))]
\]
Note that $\frac{\diff^2}{\diff\alpha^2}\expec[U(W)]<0$, hence setting the first order derivative to 0 always yields maxima $\alpha^\ast$, although there is no guarantee that such $\alpha^\ast\in[0,1]$.
\clearpage
\section{Mean-Variance Analysis}
In this chapter, the accumulation function is constant 1. Therefore, we do not consider time value.
\subsection{Return and Risk of Asset}
\begin{definition}[Rate of Return]
\hfill\\\normalfont Asset is a tradable financial instruments. We denote that each asset is traded over one time period, from $t=0$(initial) to $t=1$(end-of-period).\\
If $W_0$ invested in an asset at time $t=0$ is worth a \textbf{random} amount of $W_1$ at time $t=1$, then the \textbf{rate of return} of the asset, denoted by $r$, is a \textbf{random variable} given by
\[
r=\frac{W_1-W_0}{W_0} = \frac{W_1}{W_0}-1
\]
Equivalently, $W_1 = W_0(1+r)$.

The rate of return can also be defined in terms of the initial and end-of-period prices of the asset. Let $P_0$ be the price at $t=0$ and $P_1$ be the \textbf{random} price at $t=1$. Then
\[
r=\frac{P_1-P_0}{P_0} = \frac{P_1}{P_0}-1
\]
Equivalently, $P_1 = P_0(1+r)$,
\end{definition}
\begin{definition}[Risk of Asset]
\hfill\\\normalfont The standard deviation, $\sigma_i = \sqrt{\var(r_i)}$, of the rate of return of asset $i$, is a measure of the risk of asset $i$.
\[
\sigma_i = \sqrt{\var(r_i)}=\sqrt{\expec[(r_i-\expec(r_i))^2]} = \sqrt{\expec[r_i^2-E(r_i)^2]}
\]
\end{definition}
\begin{definition}[Correlation of Returns]
\hfill\\\normalfont A statistical measure of the association of the returns of two assets, $i$ and $j$, is the covariance $\sigma_{i,j} = \cov(r_i,r_j)$.
\[
\sigma_{i,j} = \cov(r_i,r_j) =\expec[(r_i-\expec(r_i))(r_j-\expec(r_j))] = \expec[r_ir_j]-\expec(r_i)\expec(r_j) = \expec[r_i(r_j-\expec(r_j))]
\]
A standardised measure is the correlation coefficient defined by
\[
\rho_{i,j} = \frac{\sigma_{i,j}}{\sigma_i\sigma_j}
\]
It can be shown that $|\rho_{i,j}|\leq 1$.
\end{definition}
\begin{definition}[Short Selling]
\hfill\\\normalfont Short selling of an asset refers to one borrowing a certain number of units of the asset from the lender at $t=0$ and seems them immediately to receive an amount $W_0$. At some preagreed date $t=1$, the short seller will buy the same number of units of the asset for an amount $W_1$ and return the asset to the lender.\\
The borrower will make a profit of $W_0-W_1$ which is positive if and only if the value of the asset falls.
\end{definition}
Obviously, the loss can be unlimited but the gain is bounded above by $W_0$.
\subsection{Portfolio Mean and Variance}
At time $t=0$, an individual invests in $n$ assets in such a way that a fraction $w_i$ of his investment capital is invested in asset $i$. It is possible that $w_i<0$, which means the individual short sells asset $i$. \\We call the vector $\mathbf{w} = (w_1,w_2,\ldots, w_n)^T$ the individual's \textbf{portfolio weight vector}, or simply \textbf{portfolio}. \\It is assumed that
\[
\sum_{i=1}^n w_i = 1
\] 
We will then have its final wealth $W_1 = \sum_{i=1}^n w_iW_0(1+r_i)$.\\
The rate of return $r_p$ of the portfolio is related to the rate of return of individual assets, $r_1$, by
\[
r_p = \sum_{i=1}^n w_ir_i
\]
It follows that the expected rate of return of the portfolio, or \textbf{portfolio mean}, is
\[
\mu_p = \expec(r_p) = \sum_{i=1}^n w_i\mu_i = \mathbf{w}^T\mu
\]
where 
\[
\mathbf{\mu} = (\mu_1,\ldots, \mu_n)^T
\]
is the vector of expected rates of return of the assets $(r_1,\ldots, r_n)$ respectively. This vector is called \textbf{mean vector} for simplicity.\\
The variance of rate of return of portfolio $\var(r_p)$, or simply \textbf{portfolio variance}, of $\mathbf{w}$, is
\begin{align*}
\sigma_p^2 &= \var(r_p) = \cov(\sum_{i=1}^n w_ir_i,\sum_{j=1}^n w_jr_j )\\
&=\sum_{i=1}^n\sum_{j=1}^n w_iw_j\cov(r_i,r_j)\\
&=\sum_{i=1}^n\sum_{k=1}^n \sigma_{ij}\\
&=\mathbf{w}^T\mathbf{C}\mathbf{w}\text{ in matrix notation}
\end{align*}
where
\[
\mathbf{C} = \begin{pmatrix}
\sigma_{11}&\cdots&\sigma_{1n}\\
\vdots&\ddots&\vdots\\
\sigma_{n1}&\cdots&\sigma{nn}\end{pmatrix}
\]
is known as the \textbf{covariance matrix} of the random vector $\mathbf{r} = (r_1,\cdots, r_n)$.
We also have, by noting $\sigma_{ii} = \var(r_i) := \sigma_i^2$ and $\sigma_{ij} = \sigma_{ji}$,
\[
\var(r_p) = \sum_{i=1}^n w_i^2 \sigma_i^2 + 2\sum_{i=1}^n\sum_{j<i}^n w_iw_j\sigma_{ij}
\]
\subsubsection{Diversification}
Let $\overline{\sigma}^2$ and $\overline{\phi}$ be the average variance and average covariance of an $n$ assets, that is
\[
\overline{\sigma}^2 = \frac{1}{n}\sum_{i=1}^n\sigma_i^2\;\;\;\text{ and }\;\;\;\overline{\phi}=\frac{1}{n(n-1)}\sum_{i=1}^n\sum_{\substack{j=1\\j\neq i}}^n
\] 
Suppose that $\overline{\sigma}^2\to\sigma^2$ and $\overline{\phi}\to\phi$ as $n\to\infty$, then for an equally weighted portfolio, we have
\[
\sigma_p^2 = \frac{1}{n^2}\sum_{i=1}^n \sigma_i^2 +\frac{1}{n^2}\sum_{i=1}^n\sum_{\substack{j=1\\j\neq i}}^n \sigma_{ij}\to\phi
\]
Therefore, there is limitation of diversification as a tool to reduce portfolio risk.\\While the asset specific risk $\overline{\sigma}^2$ can be driven to zero, the market wide risk, $\overline{\phi}$ cannot be eliminated even if one holds infinitely many assets.
\subsection{Portfolio of Two Assets}
Consider a portfolio with weight vector $\mathbf{w} = \begin{pmatrix}\alpha&1-\alpha\end{pmatrix}^t$ of two assets. The portfolio mean is
\[
\mu_p = \alpha\mu_1+(1-\alpha)\mu_2
\]
and
\begin{align*}
\sigma_p^2 &= \alpha^2\sigma_1^2 +(1-\alpha)^2\sigma_2^2+2\alpha(1-\alpha)\sigma_{12} = \alpha^2\sigma_1^2 +(1-\alpha)^2\sigma_2^2+2\alpha(1-\alpha)\rho_{12}\sigma_1\sigma_2 \\
&=(\sigma_1^2+\sigma_2^2-2\rho_{12}\sigma_1\sigma_2)\alpha^2+2\sigma_2(\rho_{12}\sigma_1-1)\alpha+\sigma_2^2\;\;\;\;\;(\#)
\end{align*}
\subsubsection{Global Minimum-variance Portfolio}
A risk averse individual seeks a portfolio with the \textit{smallest} risk. He will thus seek the optimal value of $\alpha$ that minimises $\sigma_p^2$.\\From the above equation $(\#), $, $\sigma_p^2$ admits a parabola concave upwards, and the minimum portfolio variance $\sigma_p^2$ occurs when
\[
\alpha=\alpha^\ast = \frac{\sigma_2(\sigma_2-\rho_{12}\sigma_1)}{\sigma_1^2+\sigma_2^2-2\rho_{12}\sigma_1\sigma_2}
\]
and the minimum portfolio variance is
\[
(\sigma_p^2)^\ast = \frac{\sigma_1^2\sigma_2^2(1-\rho_{12}^2)}{\sigma_1^2+\sigma_2^2-2\rho_{12}\sigma_1\sigma_2}
\]
The corresponding portfolio mean can then be determined from
\[
\mu_p^\ast = \alpha^\ast\mu_1+(1-\alpha^\ast)\mu_2
\]
We call the portfolio with minimum variance the global minimum-variance portfolio.
\subsubsection{Portfolio Graph}
\begin{definition}[Portfolio Graph]
\hfill\\\normalfont Portfolio graph is the graph of portfolio mean $\mu_p$ against portfolio risk $\sigma_p$.\end{definition}
From $\mu_p = \alpha\mu_1+(1-\alpha)\mu_2$, we have
\[
\alpha = \frac{\mu_p-\mu_2}{\mu_1-\mu_2}
\]
and by substituting the above equation to $(\#)$, we obtain an equation of the form
\[
\sigma_p^2 = A\mu_p^2+B\mu_p+C
\]
for some constants $A,B$ and $C$, with $A>0$.\footnote{This is due to the quadratic coefficient of $(\#)$ is greater than 0.}\\
This is an equation of a hyperbola. Rearranging
\[
\sigma_p^2 = A(\mu_p-\frac{B}{2A})^2+(C-\frac{B^2}{4A})
\]
Therefore, $\min\sigma_p^2 = C-\frac{B^2}{4A}$ at $\mu_p = -\frac{B}{2A}$. This corresponds to the global minimum-variance portfolio.\\The asymptotes of this graph are
\[
\sigma_p = \pm \sqrt{A}(\mu_p+\frac{B}{2A})
\]
or more naturally,
\[
\mu_p = \pm\frac{1}{\sqrt{A}}\sigma_p -\frac{B}{2A}
\]
When $\rho_{12}=1,-1$, this hyperbola degenerate into a pair of lines.
\end{document}