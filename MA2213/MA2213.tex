\PassOptionsToPackage{svgnames}{xcolor}
\documentclass[12pt]{article}



\usepackage[margin=1in]{geometry}  
\usepackage{graphicx}             
\usepackage{amsmath}              
\usepackage{amsfonts}              
\usepackage{framed}               
\usepackage{amssymb}
\usepackage{array}
\usepackage{amsthm}
\usepackage[nottoc]{tocbibind}
\usepackage{bm}
\usepackage{enumitem}

  \newcommand\norm[1]{\left\lVert#1\right\rVert}
\setlength{\parindent}{0cm}
\setlength{\parskip}{0em}
\newcommand{\Lim}[1]{\raisebox{0.5ex}{\scalebox{0.8}{$\displaystyle \lim_{#1}\;$}}}
\newtheorem{definition}{Definition}[section]
\newtheorem{theorem}{Theorem}[section]
\newtheorem{notation}{Notation}[section]
\theoremstyle{definition}
\newcommand\fl[1]{\text{fl}\left(#1\right)}
\DeclareMathOperator{\arcsec}{arcsec}
\DeclareMathOperator{\arccot}{arccot}
\DeclareMathOperator{\arccsc}{arccsc}
\DeclareMathOperator{\spn}{Span}
\DeclareMathOperator{\R}{\mathbf{R}}
\DeclareMathOperator{\diff}{d}
\setcounter{tocdepth}{1}
\begin{document}

\title{Revision notes - MA2213}
\author{Ma Hongqiang}
\maketitle
\tableofcontents

\clearpage
%\twocolumn
\section{Computer Arithmetic and Computational Errors}
\subsection{Number Systems}
\begin{definition}[Decimal System]
\hfill\\\normalfont The number system we are used to is the \textbf{decimal system} and the number "10" which plays an important role is called the \textbf{base} of the decimal system.
\end{definition}
In general, we can take other positive integer $N>1$ as a base. In particular, $N = 2, 8, 16$ are used in most digital computers and the systems with $N = 2,8,16$ are known as \textbf{binary}, \textbf{octal}, \textbf{hexadecimal} number systems respectively.\\
\subsubsection{Decimal to Binary}
The conversion of decimal number to binary is performed in two steps.\\
Take $(53.7)_{10}$ as an example.
\begin{enumerate}
  \item \textbf{Integer Part}. Convert decimal integers to binary by deviding by 2 successively and recording the remainders. \\The remainders, 0 or 1, are recorded by starting at the decimal point and moving away to the left. For$ (53)_{10}$, we have
  \begin{alignat*}{3}
    53/2 &= 26 &&\R 1\\
    26/2 &= 13 &&\R 0\\
    13/2 &= 6 &&\R 1\\
    6/2 &=3&&\R 0\\
    3/2 &= 1&&\R 1\\
    1/2 &= 0&&\R 1
  \end{alignat*}
  Therefore, the base 10 number 53 can be written as binary
  \[
(110101)_2 = 2^5+2^4+2^2+2^0 = 53
  \]
  \item \textbf{Fractional Part}. Convert $(0.7)_{10}$ to binary as follows: \\Multiply by 2 successfully and record the integer parts, moving away from the decimal point to the right:
  \begin{alignat*}{3}
    .7\times2 &= .4 &&+ 1\\
    .4\times2 &= .8 &&+ 0\\
    .8\times2 &= .6 &&+ 1\\
    .6\times2 &=.2&&+ 1\\
    .2\times2 &= .4&&+ 0\\
    .4\times2 &= .8&&+ 0\\
    \cdots&\cdots\cdots&& 
  \end{alignat*}
  Notice that the process repeats after four steps and will repeat infinitely exactly the same way. Therefore,
  \[
(0.7)_{10} = (.1\overline{0110})_2
  \]
  \item Hence,
  \[
(53.7)_{10} = (110101.1\overline{0110})_2
  \]
\end{enumerate}
\subsubsection{Binary to Decimal}
The conversion of binary number to decimal needs to tackle nonterminating binary numbers.\\
Take $x=(.10\overline{101})_2$ as an example.\\
Multiplying by $2^2$ shifts $x$ to
\[
y:=2^2 x = (10.\overline{101})_2 = (10)_2+(.\overline{101})_2 = (2)_{10}+(.\overline{101})_2
\]
The fractional part of $y$, i.e.,
\[
z:=(.\overline{101})_2
\]
is calculated as follows:
\[
2^3z = (101.\overline{101})_2
\]
So,
\[
(2^3-1)z= (101)_2 = (5)_{10}
\]
i.e.
\[
z = \left(\frac{5}{7}\right)_{10}
\]
Thus, 
\[
y=(2)_{10}+\left(\frac{5}{7}\right)_{10}=\left(\frac{19}{7}\right)_{10}
\]
and hence,
\[
x=\frac{y}{2^2}=\left(\frac{19}{28}\right)_{10}
\]
\begin{theorem}[Representation of Numbers]
\hfill\\\normalfont In general, any \textbf{natural numbers} $N\geq 2$ can be used as base.\\
Every positive real number $a$ has a unique representation of the form
\[
a=a_mN^m+a_{m-1}N^{m-1}+\cdots+a_1N^1+a_0N^0+a_{-1}N^{-1}+a_{-2}N^{-2}+\cdots
\]
where
\[
0\leq a_i\leq N-1,\;\;\;a_m\neq 0
\]
Equivalently,
\[
a=\sum_{i=0}^\infty a_{m-i}N^{m-i}
\]
where
\[
0<a_m<N; 0\leq a_{m-i}\leq N - 1\;\;\; \forall i\geq 1
\]
The above non-terminating representation of number $a$ is not feasible in practical computation. We onl deal with, in practice, numbers which have \textit{terminating} representation, i.e. of the following form
\[
\alpha = \sum_{i = 0}^{n-1}\alpha_{m-i}N^{m-i}
\]
where $0<\alpha_m<N$ and $0\leq \alpha_{m-i}<N$ for $i = 1,2,\ldots, n-1$.
\end{theorem}
\begin{definition}[Significant decimal digits]
\hfill\\\normalfont If $N = 10$, then the numbers $\alpha_{m-i}, 0\leq i \leq n-1$ are called \textbf{significant decimal digits} and the number $\alpha$ is said to have $n$ significant decimal digits.
\end{definition}
\subsection{Chopping and Rounding}
If the number $a$ has more significant digits than what we wish to have, then we need to replace $a$ by an approximate number $a^\ast$ which contains a smaller number of digits, say $n$. There are two ways of terminating the number $a$ to a given significant number of digits, namely \textbf{chopping} or \textbf{rounding}.
\begin{definition}[Chopping]
\hfill\\\normalfont In \textbf{chopping}, we retain only the first $n$ significant digits in the number $a$.
\end{definition}
\begin{definition}[Rounding]
\hfill\\\normalfont In \textbf{rounding}, the following rules are usually practiced:
\begin{itemize}
  \item Retain the first $n$ significant digits, and
  \item if the $(n+1)$th significant digit is less than $5$, leave the $n$th significant digit unchanged;
  \item otherwise, if the $(n+1)$th significant digit is greater or equal to $5$, add unity to the $n$th significant digit.
\end{itemize}
\end{definition}
\subsection{Floating Point Representation}
\begin{definition}[Floating Point Numbers]
\hfill\\\normalfont Number of the form
\[
\pm(0.a_1a_2a_3\ldots a_m)\times N^e, \text{with }a_1\neq 0
\]
are called \textbf{floating point numbers}. The factor $0.a_1a_2a_3\ldots a_n$ is called the \textbf{mantissa}, $e$ is called the \textbf{exponent} and $N$ is the \textbf{base}.
\end{definition}
The folating-point mode is for storing real numbers.
\subsection{Basic Concepts in Error Estimation}
\begin{definition}[Sources of Error]
\hfill\\\normalfont Numerical results are affected by many types of error.
\begin{itemize}
  \item \textbf{Error in Given Input Data}\\The input data can be \textit{result of measurements} which are inexact, or produced by some \textit{arithmetic process using round-off process}.
  \item \textbf{Round-off Errors During Computation}\\This is due to working with \textit{finite machine precision}.
  \item \textbf{Truncation Error}\\Consider the Taylor series xpansion for $\sin(x)$ about $x=0$
  \[
\sin(x) = x-\frac{x^3}{3!}+\frac{x^5}{5!}-\frac{x^7}{7!}+\ldots
  \]
which is an infinite series.\\Truncation error occurs when the \textit{infinite series is broken off after a finite number of terms}.\\In general, truncation errors are error committed when \textit{a limiting process is truncated before one has come to the limiting value}.
  \item Simplifications in the Mathematical Model
  \item Human Error
\end{itemize}
\end{definition}
\subsection{Absolute and Relative Error}
\begin{definition}[Absolute Error]
\hfill\\\normalfont Let $x^\ast$ be an approximation to $x$, then the \textbf{absolute error} in approximating $x$ is given by $|x-x^\ast|$.
\end{definition}
From the above definition, the floating point representation of $x$, $\fl{x}$ has absolute error
\[
|x-\fl{x}|
\]
\begin{definition}[Absolute Error Bound]
\hfill\\\normalfont Any non-negative number $\delta(x^\ast)$ satisfying the inequality
\[
|x-x^\ast|<\delta(x^\ast)
\]
is called an \textbf{absolute error bound}.
\end{definition}
\begin{definition}[Decimal Places]
\hfill\\\normalfont The two numbers $x$ is said to agree to $k$ \textbf{decimal places} if $k$ is the largest non-negative integer such that
\[
|x-x^\ast|\leq 0.5\times 10^{-k}
\]
\end{definition}
\begin{definition}[Relative Error]
\hfill\\\normalfont If $x^\ast$ is an approximation to $x$, then the \textbf{relative error} is defined by
\[
\frac{|x-x^\ast|}{|x|}\;\;\;\text{provided that } x\neq 0
\]
\end{definition}
For a $t$-digit machine,
\[
\frac{|x-\fl{x}|}{|x|}\leq \frac{1}{2}\times 10^{1-t}
\]
in rounding mode and
\[
\frac{|x-\fl{x}|}{|x|}\leq 10^{1-t}
\]
in chopping mode.
\subsection{Computation in Floating Point Arithmetic}
Take the example of addition/subtraction of two floating point number $0.a_1a_2a_3\ldots a_m\times 10^{N_1}\pm 0.b_1b_2b_3\ldots b_m\times 10^{N_2}$.
\begin{align*}
&0.a_1a_2a_3\ldots a_m\times 10^{N_1}\pm 0.b_1b_2b_3\ldots b_m\times 10^{N_2}\\
=&\begin{cases}
(0.a_1a_2a_3\ldots a_m\pm 0.b_1b_2b_3\ldots b_m)\times 10^{N_1}&\text{if }N_1= N_2\\
(0.a_1a_2a_3\ldots a_m 0\ldots 0\pm 0.0\ldots 0b_1b_2b_3\ldots b_m)\times 10^{N_1}&\text{if }N_1>N_2
\end{cases}
\end{align*}
\subsection{Propagation of Errors in Function Evaluation}
\subsubsection{Function of one variable}
In finding the value of $f(x)$ by approximating $f(x^\ast)$, where $x^\ast$ is the known approximating value of $x$,\\let $\delta(x^\ast)$ be an absolute error bound for $|x-x^\ast|$, i.e.,
\[
|x-x^\ast|\leq \delta(x^\ast)
\]
If the function $f(x)$ is differentiable, then by the mean-value theorem, we have
\[
f(x)-f(x^\ast)=f^\prime(p)(x-x^\ast)
\]
where $p$ is some number between $x$ and $x^\ast$.\\
Therefore, 
\[
\delta(f(x^\ast))\leq \max_{t\in\mathbf{I}}|f^\ast(t)|\delta(x^\ast)
\]
where $\mathbf{I}$ is the interval $(x,x^\ast)$ if $x<x^\ast$ or $(x^\ast,x)$.\\
Sometimes, upper bound of $f^\prime$ is unavailable. Yet, with the following assumption:
\begin{itemize}
  \item $\delta(x^\ast)$ is small
  \item $f^\prime(x^\ast)\neq 0$, and
  \item $f^\prime(t)$ is nearly constant  near $x^\ast$ (i.e., $f^\prime(t)$ does not very greatly for $t$ between $x$ and $x^\ast$)
\end{itemize}
We have $D\approx |f^\prime(x^\ast)|$, and consequently
\[
\delta(f(x^\ast))\approx|f^\prime(x^\ast)|\delta(x^\ast)
\]
\subsubsection{Function of Several Variables}
\begin{theorem}[Error bounds of function of several variables]
\hfill\\\normalfont
Generalisation suggests, if $x_i^\ast$ is an estimate of $x_i$, $1\leq i\leq n$, then with $x^\ast=(x_1^\ast,x_2^\ast,\ldots,x_n^\ast)$, we have
\[
\delta(f(x^\ast=(x_1^\ast,x_2^\ast,\ldots,x_n^\ast))\approx\left\lvert\frac{\partial f(x^\ast)}{\partial x_1}\right\rvert \delta(x_1^\ast)+\left\lvert\frac{\partial f(x^\ast)}{\partial x_2}\right\rvert \delta(x_2^\ast)+\cdots+\left\lvert\frac{\partial f(x^\ast)}{\partial x_n}\right\rvert \delta(x_n^\ast)
\]
\end{theorem}
\subsection{Catastrophic Cancellation}
Calculations involving the \textbf{subtraction} of two \textit{nearly equal} numbers can result in considerable loss of accuracy due to cancellation.\\
Let $x_1$ and $x_2$ be two nearly equal numbers and denote the error in $x_1$ and $x_2$ by $\delta(x_1)$ and $\delta(x_2)$, respectively, we have
\begin{alignat*}{3}
&&y&=x_1-x_2\\
\Rightarrow&&|\delta(y)|&\leq |\delta(x_1)|+|\delta(x_2)|\\
\Rightarrow&&\left\lvert\frac{\delta(y)}{y}\right\lvert&\leq \frac{|\delta(x_1)|+|\delta(x_2)|}{|x_1-x_2|}
\end{alignat*}
Specifically, suppose two nearly equal numbers $x_1$ and $x_2$, having the same exponent $n$ and being in the $k$ digi representations, are represented in floating point form:
\begin{align*}
x_1&=0.a_1a_2\cdots a_p\alpha_{p+1}\cdots\alpha_{k}\times 10^n\\
x_2&=0.a_1a_2\cdots a_p\beta_{p+1}\cdots\beta_{k}\times 10^n
\end{align*}
Then their difference $x_1-x_2$ in floating point representationwill have some  digits in the mantissa to be $0$, i.e.,
\[
x_1 - x_2 = \fl{\fl{x_1}-\fl{x_2}}=0.\gamma_{p+1}\cdots\gamma_{k}\underbrace{0\cdots 0}_{p\;0\text{'s}}\times 10^{n-p}
\]
So, in floating point calculation, $x_1-x_2$ has at most $k-p$ significant digits; $p$ significant digits have been lost or canceled due to the subtraction.
\subsubsection{Ways to Avoid Subtraction} 
\begin{theorem}[$\sqrt{x+\varepsilon}-\sqrt{x}$]
\hfill\\\normalfont If $|\varepsilon|\ll x$, then 
\[
\sqrt{x+\varepsilon}-\sqrt{x}=\frac{\varepsilon}{\sqrt{x+\varepsilon}+\sqrt{x}}
\]
presents a better way of calculation that reduces loss of accuracy.
\end{theorem}
\begin{theorem}[Roots of $ax^2+bx+c=0$]
\hfill\\\normalfont Consider the quadratic equation
\[
ax^2+bx+c = 0\;\;\;a\neq 0
\]
The roots of equations are
\[
x_1=\frac{-b+\sqrt{b^2-4ac}}{2a}\;\;\;\text{and}\;\;\;x_2=\frac{-b-\sqrt{b^2-4ac}}{2a}
\]
To reduce loss of accuracy, some alternative pathway include:
\begin{itemize}
\item $x_1=\frac{-2c}{b+\sqrt{b^2-4ac}}\;\;\;\text{and}\;\;\;x_2=\frac{2c}{-b+\sqrt{b^2-4ac}}$
\item $x_1+x_2 = -b$
\item $x_1x_2=c$
\end{itemize}
\end{theorem}
\begin{theorem}[Evaluation of polynomial $f(x)$]
\hfill\\\normalfont First note that $ax^n = \fl{\fl{a}\times\fl{x^n}}$ where $\fl{x^n}=\fl{\fl{x^{n-1}}\times\fl{x}}$ recursively.\\Instead of calculating $f(x) = \fl{\sum_{i=0}^n a_ix^i}$, nesting may provide better result:
\[
f(x)=\fl{(\cdots(a_nx+a_{n-1})x+\cdots)x+a_0}
\]
\end{theorem}
\subsection{Numerical Instability}
\begin{definition}[Numerical Stability]
\hfill\\\normalfont An algorithm is said to be \textbf{stable} if the effect of its errors on the final result is \textbf{negligible}.
\end{definition}
\clearpage
\section{Numerical Solution of Linear Systems of Equations}
Definitions and method of computation of matrix can be found in \texttt{MA2101.pdf}, and are therefore omitted here.
\subsection{System of Linear Equations}
In this chapter, only linear systems with \textit{unique} solution is concerned.\\
Consider the linear system
\[
Ax=b
\]
where $A\in\mathbb{M}_{n}(\mathbb{R})$ and $b\in\mathbb{R}_c^n$. If $A$ is invertible, a \textit{unique} solution $x$ exists and given by
\[
x=A^{-1}b
\]
There are two classes of method of solving $Ax=b$ numerically:
\begin{itemize}
  \item \textbf{Direct} methods
  \item \textbf{Iterative} methods
\end{itemize}
In this chapter, only direct methods are studied.\\
Two basic direct method include
\begin{itemize}
  \item Cramer's rule
  \item Gaussian Elimination
\end{itemize}
\subsubsection{Direct Method: Cramer's Rule}
\begin{definition}[Cramer's Rule]
\hfill\\\normalfont By \textbf{Cramer's rule}, the $n\times n$ linear system
\[
Ax=b
\]
has solution
\[
x_i=\frac{d_i}{d} \;\;\; i = 1,\ldots, n
\]
where $d:=\det(A)\neq 0, d_i=\det(A_i)$, $A_i$ the matrix obtained by replacing the $i$-th column of $A$ by $b$.\\
The determinant $d$ is given by \textbf{Laplace Theorem}:
\[
d=(-1)^{i+1}a_{i1}D_{i1}+(-1)^{i+2}a_{i2}D_{i2}+\cdots+(-1)^{i+n}a_{in}D_{in}
\]
where $D_{ij}$ is the determinant of the submatrix obtained from $A$ by deleting its $i$th row and $j$th column.
\end{definition}
\begin{theorem}[Computational Complexity of Cramer's Rule]
\hfill\\\normalfont Suppose it needs $m_n$ operations of multiplication to compute the determinant of a $n\times n$ matrix. Then, we have
\[
m_n=n+nm_{n-1}\;\;\;m_1=1
\]
Here, the first term $n$ is results from multiplication of $a_{ij}$ and $D_{ij}$.\\
Thus,  
\begin{align*}
m_n &=n+nm_{n-1}=n+n[(n-1)+(n-1)m_{n-2}]\\
&=n+n(n-1)+n(n-1)(n-2)+\cdots+n(n-1)\cdots 3\cdot2\\
&>n! 
\end{align*}
Hence, in order to solve an $n\times n$ linear system by Cramer's rule and Laplace Theorem, we have do at least
\[
(n+1)n!=(n+1)!
\]
  multiplication.\\Clearly, Cramer's rule is too \textit{computationally expensive}.
\end{theorem}
Also, solving nonlinear linear system by \textbf{matrix inversion} is \textit{computationally expensive} and often leads to more \textit{inaccuracies}.
\subsection{Gaussian elimination}
\begin{definition}[Elementary Row Operations]
\hfill\\\normalfont Let $E_i$ be the $i$th equation in a linear system $Ax=b$. The three \textbf{elementary row operations} permitted to solve this linear system are:
\begin{itemize}
\item $E_i\leftarrow \lambda E_i$, where $\lambda$ is a nonzero constant.
\[
\begin{bmatrix}
1&&&&\\
&\ddots&&&\\
&&\lambda&&\\
&&&\ddots&\\
&&&&1
\end{bmatrix}\begin{bmatrix}
a_{11}&\cdots&a_{1n}\\
\vdots&&\vdots\\
a_{i1}&\cdots&a_{in}\\
\vdots&&\vdots\\
a_{n1}&\cdots&a_{nn}
\end{bmatrix}=\begin{bmatrix}
a_{11}&\cdots&a_{1n}\\
\vdots&&\vdots\\
\lambda a_{i1}&\cdots&\lambda a_{in}\\
\vdots&&\vdots\\
a_{n1}&\cdots&a_{nn}
\end{bmatrix}
\]
\item $E_i\leftarrow E_i-lE_j$, where $l$ is a nonzero constant.
\[
\begin{bmatrix}
1&&&&\\
&\ddots&&&\\
&-l&1&&\\
&&&\ddots&\\
&&&&1
\end{bmatrix}^{\footnotemark}\begin{bmatrix}
a_{11}&\cdots&a_{1n}\\
\vdots&&\vdots\\
a_{i1}&\cdots&a_{in}\\
\vdots&&\vdots\\
a_{n1}&\cdots&a_{nn}
\end{bmatrix}=\begin{bmatrix}
a_{11}&\cdots&a_{1n}\\
\vdots&&\vdots\\
 a_{i1}-la_{j1}&\cdots&a_{in}-la_{jn}\\
\vdots&&\vdots\\
a_{n1}&\cdots&a_{nn}
\end{bmatrix}
\]\footnotetext{$-l$ is at $i$th row and $j$th column of the leftmost matrix.}
\item $E_i\leftrightarrow E_j$
\[
\begin{bmatrix}
1&&&&\\
&\ddots&&1&\\
&&1&&\\
&1&&\ddots&\\
&&&&1
\end{bmatrix}\begin{bmatrix}
a_{11}&\cdots&a_{1n}\\
a_{i1}&\cdots&a_{in}\\
\vdots&\cdots&\vdots\\
a_{j1}&\cdots&a_{jn}\\
a_{n1}&\cdots&a_{nn}
\end{bmatrix}=\begin{bmatrix}
a_{11}&\cdots&a_{1n}\\
a_{j1}&\cdots&a_{jn}\\
\vdots&\cdots&\vdots\\
a_{i1}&\cdots&a_{in}\\
a_{n1}&\cdots&a_{nn}
\end{bmatrix}
\]
\end{itemize}
\end{definition}
By a sequence of the above operations, a linear system can be transformed to a more easily solved linear system with the \textbf{same} set of solutions.\\
The Gaussian elimination for solving $Ax=b$ involves 2 stages/
\begin{itemize}
\item A forward course where a sequence of elementary row operations are applied to $\mathbf{A}$ to reduce it to an upper triangular form $\mathbf{U}$.
\item Backward substitution process
\end{itemize}
\begin{theorem}[General method of Gaussian Elimination on $\mathbf{A}\mathbf{x}=\mathbf{B}$]
\hfill\\\normalfont Let the original system be denoted by 
\[
\mathbf{A}^{(0)}\mathbf{x}=\mathbf{b}^{(0)}
\]
where $\mathbf{A}^{(0)} = \mathbf{A} = (a_{ij}^{(0)})$, $\mathbf{b}^{(0)}=\mathbf{b} = \begin{bmatrix}
b_1^{(0)}\\
\vdots\\
b_i^{(0)}\\
\vdots\\
b_n^{(0)}
\end{bmatrix}$\\
Step 1: If $a_{11}^{(0)}\neq 0$, let
\begin{alignat*}{23}
l_{i,1}&=\frac{a^{(0)}_{i,1}}{a^{(0)}_{1,1}}\footnotemark&&\;\;\;\;\;i = 2,3,\ldots, n\\
a^{(1)}_{i,j}&=a^{(0)}_{i,j}-l_{i,1}\times a^{(0)}_{1,j}\footnotemark&&\;\;\;\;\;i,j = 2,3,\ldots,n\\
b^{(1)}_{i}&=b^{(0)}_i-l_{i,1}\times b^{(0)}_1&&\;\;\;\;\;i = 2,\ldots,n
\end{alignat*}
\footnotetext{In general, $l_{p,q}$ is used in the operations on $q$th column, and denotes the ratio between the entry on the $p$th row($p>q$),$a^{(q-1)}_{p,q}$, to the $q$th diagonal entry, $a^{(q-1)}_{q,q}$.}
\footnotetext{In general, $a_{i,j}^{(k)}$,$(i<j)$ is obtained by operations on the $k$th column, and is the result of zeroing the $a^{(k)}_{i,k}$ entry($i>k$), so it equals the previous entry $a_{i,j}^{(k-1)}$ minus $l_{i,k}$ times the entry on the $k$th row $a^{(k-1)}_{k,j}$.}
Then we obtain the equivalent system
\[
\mathbf{A}^{(1)}\mathbf{x}=\mathbf{b}^{(1)}
\]
where
\[
\mathbf{A}^{(1)}=\begin{bmatrix}
a_{11}^{(0)}&a_{12}^{(0)}&\cdots&a_{1n}^{(0)}\\
0&a_{22}^{(1)}&\cdots&a_{2n}^{(1)}\\
0&\vdots&\ddots&\vdots\\
0&a_{n2}^{(1)}&\cdots&a_{nn}^{(1)}\end{bmatrix},\;\;\; \mathbf{b}^{(1)} = \begin{bmatrix}
b_1^{(0)}\\
b_2^{(1)}\\
\vdots\\
b_n^{(1)}
\end{bmatrix}
\]
Step 2: If $a_{22}^{(1)}\neq 0$, repeat Step 1 to eliminate $\mathbf{x}_2$ from row 3 to $n$.\\After $k$ steps, we obtain the equivalent system
\[
\mathbf{A}^{(k)}\mathbf{x}=\mathbf{b}^{(k)}
\]
where $\mathbf{A}^{(k)}$ takes the form
\[
\mathbf{A}^{(k)}=\begin{bmatrix}
a_{11}^{(0)}&\cdots       &\cdots&\cdots            &\cdots&a_{1n}^{(0)}\\
            &a_{22}^{(1)} &\cdots&\cdots            &\cdots&a_{2n}^{(1)}\\
            &             &\ddots&\vdots            &\cdots&\cdots      \\
            &             &      &a_{k+1,k+1}^{(k)} &\cdots&a_{k+1,n}^{(k)}\\
            &             &      &\vdots            &\ddots&\vdots\\
            &             &      &a_{n,k+1}^{(k)} &\cdots&a_{n,n}^{(k)}
            \end{bmatrix}
\]
If $a_{k+1,k+1}^{(k)}\neq 0$, then let
\begin{align*}
l_{i,k+1}&=\frac{a^{(k)}_{i,k+1}}{a^{(k)}_{k+1,k+1}}\\
a^{(k+1)}_{i,j}&=a^{(k)}_{i,j}-l_{i,k+1}\times a^{(k)}_{k+1,j}\\
b_i^{(k+1)}&=b_i^{(k)}-l_{i,k+1}b^{(k)}_{k+1}
\end{align*}
Finally, after at most $n-1$ steps, the system becomes
\[
\mathbf{A}^{(n-1)}\mathbf{x}=\mathbf{b}^{(n-1)}
\]
which is an upper triangular system, where
\[
\mathbf{U}=\mathbf{A}^{(n-1)}=\begin{bmatrix}
a_{11}^{(0)}&a_{12}^{(0)}&\cdots&a_{1n}^{(0)}\\
 &a_{22}^{(1)}&\cdots&a_{2n}^{(1)}\\
 & &\ddots&\vdots\\
&&&a_{nn}^{(n-1)}\end{bmatrix}
\]and\[
\mathbf{b}^{(n-1)} = \begin{bmatrix}
b_1^{(0)}\\
b_2^{(1)}\\
\vdots\\
b_n^{(n)}
\end{bmatrix}
\]
Step 3: The solution vector $\mathbf{x}$ is solved by back substitution:
\[
x_n = \frac{b_n^{(n-1)}}{a_{nn}^{(n-1)}}
\]
and for $k = n-1, n-2,\ldots, 1$,
\[
x_k = \frac{1}{a_{kk}^{(k-1)}}\left(b_k^{(k-1)}-\sum_{j=k+1}^n a_{kj}^{(k-1)}x_j\right)
\]
The above algorithm is known as \textbf{Gaussian elimination}.
\end{theorem}
\subsection{Triangular Factorisation: $\mathbf{A}=\mathbf{LU}$}
We note, in the $(k+1)$th step of above Gaussian elimination, all rows below the $(k+1)$th row are minused by a multiple of the $(k+1)$th row. This operation can be viewed by a collection of elementary row operation 2 and represented by
\[
\mathbf{A}^{(k+1)}=\mathbf{L}^{(k)}\mathbf{A}^{(k)}
\]
where
\[
\mathbf{L}^{(k)}=\begin{bmatrix}
1& & & & &\\
 &\ddots&&&&\\
 &&1&&&\\
&&-l_{k+2,k+1}&\ddots&&\\
&&\vdots&&\ddots&\\
&&-l_{n,k+1}&&&1
\end{bmatrix}
\]
This relation gives
\[
\mathbf{U}=\mathbf{A}^{(n-1)}=\mathbf{L}^{(n-2)}\mathbf{L}^{(n-3)}\cdots\mathbf{L}^{(0)}\mathbf{A}^{(0)}
\]
Also, note the inverse of $\mathbf{L}^{(k)}$ is
\[
(\mathbf{L}^{(k)})^{-1}=\begin{bmatrix}
1& & & & &\\
 &\ddots&&&&\\
 &&1&&&\\
&&l_{k+2,k+1}&\ddots&&\\
&&\vdots&&\ddots&\\
&&l_{n,k+1}&&&1
\end{bmatrix}
\]
a flip of signs of all $l_{i,k+1}$ from $\mathbf{L}^{(k)}$.\\
Therefore,
\[
(\mathbf{L}^{(0)})^{-1}(\mathbf{L}^{(1)})^{-1}\cdots (\mathbf{L}^{(n-2)})^{-1} \mathbf{U}=\mathbf{A}
\]
Note that $(\mathbf{L}^{(0)})^{-1}(\mathbf{L}^{(1)})^{-1}\cdots (\mathbf{L}^{(n-2)})^{-1}$ is upper triangular and is defined as $\mathbf{L}$:
\[
(\mathbf{L}^{(0)})^{-1}(\mathbf{L}^{(1)})^{-1}\cdots (\mathbf{L}^{(n-2)})^{-1}=\begin{bmatrix}
1&&&&&\\
l_{21}&\ddots&&&&\\
\vdots&&1&&&\\
l_{k+2,1}&\cdots&l_{k+2,k+1}&\ddots&&\\
\vdots&&\vdots&&\ddots&\\
l_{n,1}&\cdots&l_{n,k+1}&\cdots&l_{n,n-1}&1
\end{bmatrix}:=\mathbf{L}
\]
Hence, $\mathbf{A}=\mathbf{L}\mathbf{U}$, where $\mathbf{U}=\mathbf{A}^{(n-1)}$ and $\mathbf{L}$ is a lower triangular matrix.
\begin{theorem}\normalfont If $\mathbf{U}$ and $\mathbf{L}$ are the upper and lower triangular matrices defined above, then $\mathbf{A}=\mathbf{L}\mathbf{U}$.
\end{theorem}
\subsection{Compact Forms of Gaussian Elimination}
If $\mathbf{A}$ admits an $\mathbf{LU}$ decomposition, we can solve the system of equations $\mathbf{LUx}=\mathbf{b}$ in two stages.
\begin{itemize}
  \item Set $\mathbf{z}:=\mathbf{Ux}$. Solve $\mathbf{Lz}=\mathbf{b}$ for $\mathbf{z}$.
  \item Solve $\mathbf{Ux}=\mathbf{z}$ for $\mathbf{x}$
\end{itemize}
\begin{theorem}[Existence and Uniqueness of $\mathbf{LU}$ decomposition]
\hfill\\\normalfont Let $\mathbf{A}$ be an $n\times n$ matrix and $\mathbf{A}^{(k)}$ be the $k\times k$ matrix formed from the first $k$ rows and columns of $\mathbf{A}$. If $\det(\mathbf{A}^{(k)})\neq 0 \forall k = 1,2,\ldots, n-1$, then there exists a unique lower triangular matrix $\mathbf{L}=(l_{ij})$ with $l_{ii}=1 \forall i=1,2\ldots,n$ and a unique upper triangular matrix $\mathbf{U}=(u_{ij})$ such that $\mathbf{A}=\mathbf{LU}$.
\end{theorem}
\subsection{Compact Forms of Gaussian Elimination}
The $\mathbf{LU}$ decomposition can be calculated directly by Gaussian Elimination.\\
Given that $mathbf{A}$ admits a $\mathbf{LU}$ decomposition $\mathbf{A}=\mathbf{LU}$, we have, entry-wise
\[
a_{ij} = \sum_{k=1}^n l_{ik}u_{kj}\;\;\;i,j = 1,2,\ldots,n
\]
Note that the above system has $n^2$ equations and $n^2+n$ unknowns $l_{ik}, i\geq k$ and $u_{kj},k\leq j$. Thus $n$ unknowsn may be set arbitrarily.
\begin{theorem}[Doolittle Method]
\hfill\\\normalfont For Doolittle Method, set diagonal entries of $\mathbf{L}$, $l_{kk}:=1, k = 1,2,\ldots n$ and assume that $u_{kk}\neq 0,\forall k \in[1,n]\cap\mathbb{Z}$.\\
The sequence of calculation is as follows:
\begin{enumerate}
  \item First row of $\mathbf{U}$: $u_{1k}, 1\leq k\leq n$
  \item First column of $\mathbf{L}$: $l_{k1}, 2\leq k\leq n$\footnote{if $k = 1, l_{k1}=l_{11}=1$ by definition}
  \item Second row of $\mathbf{U}$: $u_{2k}, 2\leq k\leq n$.
  \item[] $\vdots$
\end{enumerate}
Obviously, when calculating $k$th row of $\mathbf{U}$, row 1 to $k-1$ of $\mathbf{U}$ and column 1 to $k-1$ of $\mathbf{L}$ will be known.\\
Also, when calculating $k$th row of $\mathbf{L}$, row 1 to $k$ of $\mathbf{U}$ and column 1 to $k-1$ of $\mathbf{L}$ will be known.\\
Suppose $u_{kj}, (j\geq k)$ is of concern, we have
\[
a_{kj}=\sum_{r=1}^k l_{kr}u_{rj}
\]
Substitute in $l_{kk}=1$ and rearranging,
\[
u_{kj} = a_{kj} -\sum_{r=1}^{k-1}l_{kr}u_{rj}
\]
Afterwards, suppose $l_{ik}, (i > k)$ is of concern, we have
\[
a_{ik} = \sum_{r=1}^k l_{ir}u_{rk}
\]
Rearranging,
\[
a_{ik}=\sum_{r=1}^{k-1} l_{ir}u_{rk}+l_{ik}u_{kk}
\]
Therefore,
\[
l_{ik}=\frac{a_{ik}-\sum_{r=1}^{k-1} l_{ir}u_{rk}}{u_{kk}}
\]
The above two equation can be interleaved to obtain $\mathbf{L}$ and $\mathbf{U}$.
\end{theorem}
\begin{theorem}[Crout Method]
\hfill\\\normalfont For Crout Method, set diagonal entries of $\mathbf{U}$, $u_{ii}=1, k = 1,2,\ldots, n$ and assume that $l_{kk}\neq 0 \forall k\in[1,n]\cap\mathbb{Z}$.\\
The sequence of calculation is as follows:
\begin{enumerate}
  \item First column of $\mathbf{L}$: $l_{k1}, 1\leq k\leq n$
  \item First row of $\mathbf{U}$: $u_{1k}, 2\leq k\leq n$
  \item Second column of $\mathbf{L}$: $l_{k2}, 2\leq k\leq n$
  \item[] $\vdots$
\end{enumerate}
Obviously, when calculating $k$th column of $L$, column 1 to $k-1$ of $\mathbf{L}$ and row 1 to $k-1$ of $\mathbf{U}$ will be known.\\
Also, when calculating $k$th row of $\mathbf{U}$, column 1 to $k$ of $\mathbf{L}$ and row 1 to $k-1$ of $\mathbf{U}$ will be known.\\
Suppose $l_{ik}, (i\geq k)$ is of concern, we have
\[
a_{ik} = \sum_{r = 1}^k l_{ir}u_{rk}
\]
Substitute in $u_{kk}=1$ and rearranging,
\[
l_{ik} = a_{ik} - \sum_{r = 1}^{k-1}l_{ir}u_{rk}
\]
Afterwards, suppose $u_{kj}, (j\geq k+1)$ is of concern, we have
\[
a_{kj} = \sum_{r = 1}^k l_{kr}u_{rj}
\]
Rearranging, 
\[
u_{kj} = \frac{a_{kj}-\sum_{r=1}^{k-1}l_{kr}u_{rj}}{l_{kk}}
\]
The above two equation can be interleaved to obtain $\mathbf{L}$ and $\mathbf{U}$.
\end{theorem}
\subsection{$\mathbf{LU}$ Decomposition for Tridiagonal Matrices}
\begin{theorem}\hfill\\\normalfont
Let $\mathbf{A}$ be a \textbf{tridiagonal matrices} 
\[
\mathbf{A}=\begin{bmatrix}
a_1&c_1&&&&\\
b_2&a_2&c_2&&&\\
&\ddots&\ddots&\ddots&&\\
&&\ddots&\ddots&\ddots&\\
&&&b_{n-1}&a_{n-1}&c_{n-1}\\
&&&&b_n&a_n
\end{bmatrix}
\]
If an $\mathbf{LU}$ decomposition exists for $\mathbf{A}$, then
\[
\mathbf{A}=\mathbf{LU}
\]
where
\[
\mathbf{L}=\begin{bmatrix}
1&&&&&\\
\beta_2&1&&&&\\
&\ddots&\ddots&\ddots&&\\
&&\ddots&\ddots&\ddots&\\
&&&\beta_{n-1}&1&\\
&&&&\beta_n&1
\end{bmatrix}\;\;\;\text{and}\;\;\;\mathbf{U}=\begin{bmatrix}
\alpha_1&c_1&&&&\\
&\alpha_2&c_2&&&\\
&\ddots&\ddots&\ddots&&\\
&&\ddots&\ddots&\ddots&\\
&&&&\alpha_{n-1}&c_{n-1}\\
&&&&&\alpha_n
\end{bmatrix}
\]
By applying Doolittle's method, it can be shown that
\[
\alpha_1 = a_1
\]
and for $k = 2, 3,\ldots, n$,
\begin{align*}
\beta_k &=\frac{b_k}{\alpha_{k-1}}\\
\alpha_k&=a_k - \beta_k c_{k-1}
\end{align*}
\end{theorem}
\begin{theorem}[Thomas's Algorithm]
\hfill\\\normalfont If $\mathbf{Ax}=\mathbf{g}$ and define $\mathbf{Ux} = h$ where $\mathbf{x}:=(x_1,x_2,\ldots, x_n)^\text{T}$, $\mathbf{g}=(g_1,g_2,\ldots, g_n)^\text{T}$ and $\mathbf{h}=(h_1,h_2,\ldots, h_n)^\text{T}$. 
\begin{align*}
h_1&=g_1\\
h_i&=g_i-\beta_ih_{i-1},\;\;\; i = 2, 3, \ldots, n
\end{align*}
and
\begin{align*}
x_n &=\frac{h_n}{\alpha_n}\\
x_i &=\frac{h_i-c_ix_{x+1}}{\alpha_i}
\end{align*}
This method is known as \textbf{Thomas algorithm}.
\end{theorem}
\subsection{Pivoting Strategies}
\subsubsection{Partial Pivoting}
\begin{definition}[Partial Pivoting]\hfill\\\normalfont
At the $(k+1)$th step of the Gaussian Elimination process, $k = 0,1,\ldots, n-2$, choose the element having maximum absolute value in the $(k+1)$th column of $\mathbf{A}^{(k)}$ that lies on or below the diagonal so that
\[
|a_{s,k+1}^{(k)}|=\max_{i}|a_{i,k+1}^{(k)}|\;\;\;k+1\leq i\leq n
\]
and interchange row $k+1$ with row $s$.
\end{definition}
\subsubsection{Scaled Partial Pivoting}
\begin{definition}[Scaled Partial Pivoting]\hfill\\\normalfont
In the beginning, calculate $s_i$ for all row $i$, where
\[
s_i = \max_{1\leq j\leq n}\{|a_{ij}|\} , i = 1, \ldots, n
\]
At the $(k+1)$th step of the Gaussian Elimination process, $k = 0,1,\ldots, n-2$, choose $r$th row of $\mathbf{A}^{(k)}$ that lies on or below the diagonal where $r$ is determined by
\[
\frac{|a_{r,k+1}^{(k)}|}{s_r}=\max_{i}\frac{|a_{i,k+1}^{(k)}|}{s_i}\;\;\;k+1\leq i\leq n
\]
and exchange row $k+1$ with row $r$ and $s_{k+1}$ with $s_{r}$.
\end{definition}
\clearpage
\section{Error of Approximation}
\begin{definition}[Error]
\hfill\\\normalfont Given $n+1$ data values $(x_0,f_0), (x_1,f_1),\ldots, (x_n,f_n)$, we may define the error of approximation, $E$, by
\[
E=\sum_{i=0}^n|p(x_i)-f_i|
\]
If we take $p(x)$ to be a polynomial of degree $n$ then by choosing
\[
p(x_i)=f_i, i= 0,1,\ldots, n
\]
we can make $E = 0$. The function $p(x)$ is the \textbf{Lagrange interpolating polynomial}.\\
Alternatively, when $n$ is large, we may fit a polynomial of degree $k$ where $k\ll n$. We use
\[
E=\sum_{i=0}^n[p(x_i)-f_i]^2
\]
By minimising $E$ we achieve a \textbf{least square fit}.\\
If the function $f(x)$ is continuous specified on the interval $[a,b]$, we may define the error $E$ in approximating $f(x)$ by $p(x)$ as
\[
E=\max_{a\leq x \leq b}|p(x)-f(x)|
\]
It is common in practice to take $p(x)$ as a polynomial and minimise $E$ with respect to variations in the coefficients. This generate the \textbf{minimax} polynomial approximation.
\end{definition}
The following theorem justifies the choice of polynomial for approximation
\begin{theorem}[Weierstrass Approximation Theorem]
\hfill\\\normalfont Suppose $f$ is defined and continuous on $[a,b]$. For each $\varepsilon>0$ there exists a polynomial $P(x)$, defined on $[a,b]$, with property that
\[
|f(x)-P(x)|<\varepsilon\;\;\;\forall x\in[a,b]
\]
\end{theorem}
\subsection{Least Square Approximation}
\subsubsection{Discrete Data}
Given a set of $m+1$ discrete data points $(x_0,f(x_0)), (x_1,f(x_0)),\ldots (x_m,f(x_m))$, each carrying weight $w_0,w_1,\ldots, w_m$ respectively. \\
The task is to find an approximating polynomial function\footnote{The subscript $n$ denotes the power of the polynomial; thus the polynomial $p_n$ has $n+1$ unknwons}
\[
p_n(x;a_0,a_1,\ldots, a_n):=a_0+a_1x+\cdots+a_nx^n
\] 
$(m>n)$ such that
\[
E(a_0,a_1,\ldots, a_n)=\sum_{i=0}^m w_i[f(x_i)-p_n(x_i; a_0,a_1,\ldots, a_n)]^2
\]
is minimised with respect to the parameters $a_0, a_1,\ldots, a_n$.\footnote{$E$ can be understood as a weighted sum of squares of residue for each data.}\\
We require
\[
\frac{\partial E}{\partial a_j} = 9 \forall j = 0,1,\ldots, n
\]
Therefore, by differentiating with $a_j$,
\[
\sum_{i=0}^m w_i[f(x_i)-p_n(x_i; a_0,a_1,\ldots, a_n)]\frac{\partial}{\partial a_j}\left(-p_n(x_i;a_0,a_1,\ldots, a_n)\right)\;\;\;\forall j = 0,1,\ldots, n
\]
Rearranging, we have
\[
\sum_{i=0}^mw_i\underbrace{(a_0+a_1x_i+\cdots+a_nx_i^n)}_{p_n(x_i; a_0,a_1,\ldots, a_n)}x_i^j   =\sum_{i=0}^m w_if(x_i)x_i^j\;\;\;\forall j = 0,1,\ldots, n
\]
There is $n+1$ unknowns, and $n+1$ equations, so the linear system is\footnote{the $j$th column corresponding to the equation of specific $j$}
\[
\begin{bmatrix}
\sum_{i=0}^m w_ix_i^0&\sum_{i=0}^m w_ix_i^1&\cdots \sum_{i=0}^m w_ix_i^n\\
\sum_{i=0}^m w_ix_i^1&\sum_{i=0}^m w_ix_i^2&\cdots \sum_{i=0}^m w_ix_i^{n+1}\\
\vdots&\vdots&\ddots&\vdots\\
\sum_{i=0}^m w_ix_i^n&\sum_{i=0}^m w_ix_i^{n+1}&\cdots \sum_{i=0}^m w_ix_i^{2n}\\
\end{bmatrix}\begin{bmatrix}
a_0\\a_1\\\vdots\\a_n\end{bmatrix}=\begin{bmatrix}
\sum_{i=0}^m w_if(x_i)x_i^0\\\sum_{i=0}^m w_if(x_i)x_i^1\\\vdots\\\sum_{i=0}^m w_if(x_i)x_i^n\end{bmatrix}
\]
\subsubsection{Continuous Function}
Given a continuous function $f(x)$ defined on an interval $[a,b]$ with each value $x\in[a,b]$ associated with a weight $w(x)$. \\
The task is to find an approximating polynomial function $p_n(x;a_0,a_1,\ldots, a_n)$ such that
\[
E(a_0,a_1,\ldots, a_n)=\int_a^b w(x)[f(x)-p_n(x;a_0,a_1,\ldots, a_n)]^2\diff x
\]
is minimised with respect to the parameters $a_0,a_1,\ldots, a_n$.\\
The necessary condition sfor $E(a_0,a_1,\ldots, a_n)$ to be minimum are
\[
\frac{\partial E}{\partial a_j}=0 \forall j = 0,1,\ldots, n
\]
Differentiating with respect to $a_j$,
\[
\frac{\diff E}{\diff a_j} = -2\int_a^b w(x)[f(x)-p(x;a_0,a_1,\ldots, a_n)]\frac{\partial}{\partial a_j}(p_n(x;a_0,a_1,\ldots, a_n))\diff x = 0
\]
Rearranging,\footnote{$\frac{\partial}{\partial a_j}(p_n(x;a_0,a_1,\ldots, a_n))=x^j$ is obvious.}
\[
\int_a^b w(x)(a_0+a_1x+\cdots+a_nx^n)x^j\diff x = \int_a^b w(x)f(x)x^j\diff x\;\;\;\forall j = 0,1,\ldots, n
\]
\[
\sum_{k = 0}^n \int_a^b w(x)x^{k+j}\diff x = \int_a^b w(x)f(x)x^j\diff x\;\;\;\forall j = 0,1,\ldots, n
\]
Similarly, a matrix equation can be formed from this linear system of $n+1$ variables and $n+1$ equations. However, this matrix is \textbf{ill-conditioned}. 
\subsection{Lagrange Interpolation}
\begin{definition}\hfill\\\normalfont
Given a set of $n+1$ data points $(x_0,f_0),(x_1,f_1),\ldots,(x_n,f_n)$, we seek to find a polynomial $p_n(x)$ of degree $n$ which passes through each of the given points, i.e.
\[
p_n(x_i)=f_i\;\;\;i = 0,1,2,\ldots, n
\]
If we write $p_n(x)= a_0+a_1x+a_2x^2+\cdots+a_nx^n$, we have
\[
\begin{bmatrix}
1&x_0&\cdots&x_0^n\\
1&x_1&\cdots&x_1^n\\
\vdots&\vdots&\ddots&\vdots\\
1&x_n&\cdots&x_n^n
\end{bmatrix}\begin{bmatrix}a_0\\a_1\\\vdots\\a_n\end{bmatrix} = \begin{bmatrix}f_0\\f_1\\\vdots\\f_n\end{bmatrix}
\]
The determinant of the $(n+1)\times(n+1)$ Vandermonde matrix is $\displaystyle\prod_{0\leq i<j\leq n}(x_j-x_i)$, so the system has a unique solution as long as $x_i, i = 1,2,\ldots, n$ are all distinct.\\
The solved polynomial is the unique Lagrange interpolating polynomial.
\end{definition}
\begin{theorem}[Computation of Lagrange Interpolating Polynomial]
\hfill\\\normalfont Write
\begin{align*}
p_n(x)&=\alpha_0(x-x_1)(x-x_2)\cdots(x-x_n)\\
&+\alpha_1(x-x_0)(x-x_2)\cdots(x-x_n)\\
&+\cdots\\
&+\alpha_n(x-x_0)(x-x_1)\cdots(x-x_{n-1})
\end{align*}
Then, we have
\[
 \alpha_i = \frac{f_i}{\displaystyle\prod_{\substack{j=0\\ j\neq i}}^n(x_i-x_j)}
\]
So,
\[
p_n(x) = \sum_{i=0}^n l_i(x)f_i
\]
where
\[
l_i(x) = \prod_{\substack{j=0\\j\neq i}}^n\frac{x-x_j}{x_i-x_j}
\]
\end{theorem}
Note that $l_i(x)$ admits the property that
\begin{align*}
l_i(x_i)&=1\\
l_i(x_j)&\neq 1 \text{ for }j\neq i
\end{align*}
\begin{theorem}[$\psi(x)$ and alternative expression of Lagrange Interpolating Polynomial]
\hfill\\\normalfont Let $\psi(x)$ denote the polynomial
\[
\psi(x)=(x-x_0)(x-x_1)\cdots(x-x_n) = \prod_{i=0}^n (x-x_i)
\]
then, by differentiating on each product term once at a time, we have
\[
\psi^\prime(x) = \sum_{i=0}^n\prod_{\substack{j=0\\j\neq i}}^n (x-x_j)
\]
and hence
\[
\psi^\prime (x_i) = (x_i-x_0)\cdots(x_i-x_{i-1})(x_i-x_{i+1})\cdots(x_i-x_n)
\]
as all other terms containing $(x-x_i)$ will vanish.\\
Thus,\footnote{Do NOT forget multiply $(x-x_i)$ while using this formula!}
\[
l_i(x)=\frac{\psi(x)}{(x-x_i)\psi^\prime(x_i)}
\]
So $p_n(x)$ may be written as
\[
p_n(x)=\sum_{i=0}^n\frac{\psi(x)}{(x-x_i)\psi^\prime(x_i)} f_i
\]
\end{theorem}
\subsection{Truncation Error of Interpolating Polynomial}
\begin{theorem}[Extended Mean Value Theorem]
\hfill\\\normalfont Suppose that
\[
a\leq x_0<x_1<\cdots<x_k\leq b
\]
\[
f(x_0)=f(x_1)=\cdots = f(x_k)=0
\]
and $f(x),f^\prime(x),\ldots, f^{(k)}(x)$ are all continuous on $[a,b]$. There is a $\xi\in(x_0,x_n)\subset(a,b)$ such that
\[
f^k(\xi)=0
\]
\end{theorem}
\begin{theorem}[Lagrange Interpolating Polynomial Error Formula]
\hfill\\\normalfont Let $f(x)\in C^{(n+1)}[a,b]$. Further, let $p_n(x)$ interpolates $f(x)$ at $(n+1)$ distinct points $x+0,x_1,\ldots, x_n\in[a,b]$. Then, for $x\in[a,b], x\neq x_i, i = 0,1,\ldots, n$,
\[
f(x)-p_n(x)=\psi(x)\frac{f^{(n+1)}(\xi)}{(n+1)!}
\]
for some $xi\in\text{Spr}\{x,x_0,x_1,\ldots,x_n\}$.\\
Here, $\text{Spr}\{x,x_0,x_1,\ldots, x_n\}$ denotes the smallest interval containing $x,x_0,x_1,\ldots,x_n$.
\end{theorem}
It follows that an upper bound for the error is
\[
\delta(f(x)):=|f(x)-p_n(x)| \leq \frac{M}{(n+1)!}|\psi(x)|
\]
where
\[
M=\max_{a\leq \xi\leq b}|f^{(n+1)}(\xi)|
\]
\subsection{Divided Differences}
\begin{definition}[Divided Differences]
\hfill\\\normalfont Let $(x_0,f(x_0)), \ldots, (x_n,f(x_n))$ be $n+1$ given data points. The \textbf{first divided difference} of $f(x)$ between $x_0$ and $x_1$, denoted by $f[x_0,x_1]$ is defined by
\[
f[x_0,x_1]:=\frac{f(x_1)-f(x_0)}{x_1-x_0}
\]
Analogously, the \textbf{second divided difference} for the triplet $(x_0,x_1,x_2)$ is defined by
\[
f[x_0,x_1,x_2]:=\frac{f[x_1,x_2]-f[x_0,x_1]}{x_2-x_0}
\]
And, the $n$th divided difference of $f(x)$ at $(n+1)$-tuple $(x_0,\ldots, x_n)$ is defined recursively in terms of the $(n-1)$th divided difference as follows:
\[
f[x_0,x_1,\ldots, x_n]:=\frac{f[x_1,\ldots, x_n]-f[x_0,\ldots, x_{n-1}]}{x_n-x_0}
\]
\end{definition}
\begin{theorem}[Divided difference of self]
\hfill\\\normalfont We define $f[x_i,x_i]$ by
\[
f[x_i,x_i] = \lim_{x_{i+1}\to x_i}f[x_i,x_{i+1}] = f^\prime(x_i)
\]
provided $f^\prime(x_i)$ exists.\\
Similarly, we define $r$th divided difference of self is
\[
f[\underbrace{x_i,\ldots, x_i}_{r x_i's}] = \frac{1}{r!}f^{(r)}(x_i)
\]
\end{theorem}
\subsection{Newton Interpolation Formula}
It is easy to see, by definition of first order divided difference, that
\[
f(x) = f(x_0)+(x-x_0)f[x,x_0]
\]
Therefore, we can obtain
\begin{align*}
f(x) &= f(x_0)+(x-x_0)f[x_0,x_1]+\cdots +(x-x_0)\cdots(x-x_{n-1})f[x_0,\ldots, x_n]+R_n(x)\\
&=f(x_0)+\sum_{i=0}^{n-1}f[x_0,\ldots, x_{i+1}]\prod_{j=0}^i (x-x_j)+R_n(x)
\end{align*}
where
\[
R_n(x) = \psi(x)f[x,x_0,x_1,\ldots, x_n]
\]
The Newton interpolating formula of these $n+1$ data points is 
\[
p_n(x) = f(x_0)+(x-x_0)f[x_0,x_1]+\cdots +(x-x_0)\cdots(x-x_{n-1})f[x_0,\ldots, x_n]
\]
To see this, we note that at $x_0, \ldots, x_n$, $R_n=0$, so $f(x) = p_n(x)$; also $p_n(x)$ is at most of degree $n$.

When expanding the formula $p_n$ to $p_{n+1}$, we note that $f = p_{n+1} = p_n + (x-x_0)\cdots(x-x_n)f[x_0,\cdots, x_{n+1}]$, so the divided difference can be easily acquired.\\

\textbf{Remark}: $f[x_0,\ldots,x_n] = f[x_{i_0},\ldots, x_{i_n}]$, where $\{i_k\}_{k=1}^n$ is a permutation of $\{k\}_{k=1}^n$.
\subsection{Cubic Spline Interpolation}
Cubic Spline interpolation is a piecewise approximation used to replace high order polynomial interpolation, latter of which has the problem of wild oscillation.
\begin{definition}[Cubic Spline Approximation]
\hfill\\\normalfont Given a function $f$ defined on $[a,b]$ and a set of data points $x_i$ with $a=x_0<x_1<\cdots<x_{n-1}<x_n=b$, a \textbf{cubic spline interpolant} to $f$ is a \textit{piecewise} function $S$ satisfying the following conditions:
\begin{enumerate}
  \item $S$ is a cubic polynomial, denoted $S_j$ on the subinterval $[x_j,x_{j+1}]$ for $j = 0,1,\ldots, n-1$.
  \item Interpolation condition: $S(x_j) = f(x_j)$.
  \item $S\in C^1(a,b)$:$S^\prime_j(x_{j+1}) = S^\prime_{j+1}(x_{j+1})\;\;\;j = 0,1,\ldots, n-2$.
  \item $S\in C^2(a,b)$:$S^{\prime\prime}_j(x_{j+1}) = S^{\prime\prime}_{j+1}(x_{j+1})\;\;\;j = 0,1,\ldots, n-2$.
  \item One of the following sets of boundary condition is satisfied
  \begin{itemize}
    \item $S_0^{\prime\prime}(x_0)=S_{n-1}^{\prime\prime}(x_n)=0$ (Natural boundary condition)
    \item Or, $S_0^\prime(x_0) = f^\prime(x_0)$ and $S_{n-1}^\prime(x_n)= f^\prime(x_n)$ (Clamped boundary condition)
  \end{itemize}
\end{enumerate}
\end{definition}
\begin{theorem}[Solving cubic spline interpolation efficiently]
\hfill\\\normalfont The following method will produce the cubic spline interpolation of $n+1$ data points by solving a linear system of maximum size $(n+1)\times(n+1)$.

Let $s_i(x)$ be a cubic polynomial on $[x_i,x_{i+1}]$. Then $s^{\prime\prime}_i(x)$ is the linear polynomial on $[x_i,x_{i+1}]$.\\Let $M_i = s^{\prime\prime}_i(x_i)$. $M_i$ will be calculated later using (3) and (5).

Using lagrange interpolation polynomial, we have
\[
s^{\prime\prime}_i(x) = \frac{x_{i+1}-x}{h_i}M_i+\frac{x-x_i}{h_i}M_{i+1}
\]
where $h_i = x_{i+1}-x_i$ the step size.\\
Integrating twice, we have
\[
s_i(x) = \frac{(x_{i+1}-x)^3}{6h_i}M_i+\frac{(x-x_i)^3}{6h_i}M_{i+1}+Ax+B
\]
for some constant $A$ and $B$.\\
Using (2), i.e.,
\[
s_i(x_i) = f_i\;\;\;\;\;s_i(x_{i+1})=f_{i+1}
\]
we have, nicely\footnote{At each substitution, one cubic power vanishes and the other degenerates to $h_i$, which simplifies the calculation.}
\begin{align*}
\frac{h_i^2}{6}M_i+Ax_i+B&=f_i\\
\frac{h_i^2}{6}M_{i+1}+Ax_{i+1}+B=f_{i+1}
\end{align*}
Solving which we have
\[
\begin{cases}
A&=\frac{f_{i+1}-f_i}{h_i}+\frac{h_i}{6}(M_i-M_{i+1})\\
B&=\frac{x_{i+1}f_i-x_if_{i+1}}{h_i}+\frac{h_i}{6}(x_iM_{i+1}-x_{i+1}M_i)
\end{cases}
\]
Therefore,
\begin{align*}
s_i(x) &= \frac{6}{h_i}[(x_{i+1}-x)^3M_i+(x-x_i)^3M_{i+1}]\\
      &-\frac{h_i}{6}[(x_{i+1}-x)M+(x-x_i)M_{i+1}]\\
      &+\frac{1}{h_i}[(x_{i+1}-x)f_i+(x-x_i)f_{i+1}]
\end{align*}
From here, differentiate $s_i(x)$, we have expression for $s_i^\prime(x)$:
\[
s_i^\prime(x) = \frac{1}{2h_i}[-(x_{i+1}-x)^2M_i+(x-x_i)^2M_{i+1}]-\frac{h_i}{6}(-M_i+M_{i+1})+\frac{1}{h_i}(-f_i+f_{i+1})
\]
Replacing the subscript $i$ by $i-1$, we have
\[
s_{i-1}^\prime(x) = \frac{1}{2h_{i-1}}[-(x_{i}-x)^2M_{i-1}+(x-x_{i-1})^2M_{i}]-\frac{h_{i-1}}{6}(-M_{i-1}+M_{i})+\frac{1}{h_{i-1}}(-f_{i-1}+f_{i})
\]
Imposing (3), i.e. $s^\prime_{i-1}(x_i) = s^\prime_{i}(x_i)$, we will have
\[
h_{i-1}M_{i-1}+2(h_{i-1}+h_i)M_i+h_iM_{i+1} = 6\frac{f_{i+1}-f_i}{h_i}-6\frac{f_i-f_{i-1}}{h_{i-1}}
\]
for $i = 1,\ldots n-1$. This equation represents a system of $n-1$ linear equations with $n+1$ unkwowns; the other 2 equation is obtained from the boundary condition.\\
\textbf{Natural boundary condition}:\\
Under natural boundary condition, $M_0 = M_n = 0 $ by definition. Then we have 
\[
\mathbf{Am} = \mathbf{b}
\] 
where 
\[
\mathbf{A} = \begin{pmatrix} 2(h_0+h_1)&h_1&&&&\\
h_1&2(h_1+h_2)&h_2&&&\\
&\ddots&\ddots&\ddots&&\\
&&h_{n-3}&2(h_{n-3}+h_{n-2})&h_{n-2}&\\
&&&h_{n-2}&2(h_{n-2}+h_{n-1})\end{pmatrix}
\]
\[
\mathbf{m} = \begin{pmatrix} M_1&M_2&\cdots&M_{n-1}\end{pmatrix}^t
\]
and
\[
\mathbf{b} = 6\times\begin{pmatrix}
\frac{f_0}{h_0}-\frac{f_1}{h_0}-\frac{f_1}{h_1}+\frac{f_2}{h_1}\\
\vdots\\
\frac{f_{i-1}}{h_{i-1}}-\frac{f_i}{h_{i-1}}-\frac{f_i}{h_i}+\frac{f_{i+1}}{h_i}\\
\vdots\\
\frac{f_{n-2}}{h_{n-2}}-\frac{f_{n-1}}{h_{n-2}}-\frac{f_{n-1}}{h_{n-1}}+\frac{f_n}{h_{n-1}}\end{pmatrix}
\]
\textbf{Clamped boundary condition}:\\
Under clamped boundary condition, $s_0^\prime(x_0) = f^\prime(x_0)$ and $s_{n-1}^\prime(x_n) = f^\prime(x_n)$, which we will have, from $s_i^\prime (x)$ equation,
\begin{align*}
2h_0M_0+h_0M_1 &= 6(-\frac{f_0}{h_0}+\frac{f_1}{h_0}-f^\prime_0)\\
h_{n-1}M_{n-1}+2h_{n-1}M_n &= 6(\frac{f_{n-1}}{h_{n-1}}-\frac{f_n}{h_{n-1}}+f^\prime_n)
\end{align*}
Therefore, we can form $\mathbf{Am}=\mathbf{b}$ accordingly, where
\[
\mathbf{A} = \begin{pmatrix} 2h_0&h_0&&&&\\
h_0&2(h_0+h_1)&h_1&&&\\
&\ddots&\ddots&\ddots&&\\
&&h_{n-2}&2(h_{n-2}+h_{n-1})&h_{n-1}&\\
&&&h_{n-1}&2h_{n-1}\end{pmatrix}
\]
\[
\mathbf{m} = \begin{pmatrix} M_0&M_1&\cdots&M_{n}\end{pmatrix}^t
\]
and
\[
\mathbf{b} = 6\times\begin{pmatrix}
-\frac{f_0}{h_0}+\frac{f_1}{h_0}-f_0^\prime\\
\frac{f_0}{h_0}-\frac{f_1}{h_0}-\frac{f_1}{h_1}+\frac{f_2}{h_1}\\
\vdots\\
\frac{f_{i-1}}{h_{i-1}}-\frac{f_i}{h_{i-1}}-\frac{f_i}{h_i}+\frac{f_{i+1}}{h_i}\\
\vdots\\
\frac{f_{n-2}}{h_{n-2}}-\frac{f_{n-1}}{h_{n-2}}-\frac{f_{n-1}}{h_{n-1}}+\frac{f_n}{h_{n-1}}\\
\frac{f_{n-1}}{h_{n-1}}-\frac{f_n}{h_{n-1}}+f_n^\prime
\end{pmatrix}
\]
\end{theorem}
\clearpage
\section{Numerical Integration}
\subsubsection{Prerequisite -- Useful Theorems}
\begin{theorem}[Intermediate Value Theorem]
\hfill\\\normalfont 
If $f(x)$ is continuous on $[a,b]$, let $m=\displaystyle{\min_{a\leq x\leq b}f(x)}$ and $M=\displaystyle{\max_{a\leq x\leq b}f(x)}$, then for any $K$ satisfying $m\leq K\leq M$, there exists $xi\in(a,b)$ such that $f(\xi)=K$.
\end{theorem}
From this theorem we have the following theorem.
\begin{theorem}[Mean Value Theorem for Integrals]
\hfill\\\normalfont Let $g(x)$ be a \textbf{non-negative} or \textbf{non-positive} integrable function on $[a,b]$. If $f(x)$ is continuous on $[a,b]$, then there is at least one point $\xi$ in $[a,b]$ such that
\[
\int_a^b f(x)g(x)\diff x = f(\xi)\int_a^bg(x)\diff x
\]
\end{theorem}
\subsection{Quadrature based on Interpolation Polynomials}
We first select a set of $(n+1)$ distinct nodes $\{x_0,\ldots, x_n\}$ from the interval $[a,b]$. The Lagrange interpolating polynomial is
\[
p_n(x)=\sum_{i=0}^n l_i(x)f(x_i)
\]
with truncation error term
\[
E_n(x) = \frac{f^{(n+1)}(\xi)}{(n+1)!}\psi(x)
\]
which admits $f(x)=p_n(x)+E_n(x)$.

We approximate $\int_a^b f(x)$ by considering the integral of $p_n(x)$, denoted by $I_n(f)$
\[
I_n(f):=\int_a^b p_n(x)\diff x = \int_a^b\sum_{i=0}^n l_i(x)f(x_i)\diff x = \sum_{i=0}^n a_if(x_i)
\]
where $a_i = \int_a^b l_i(x)$.
Therefore, the truncation error will generate the error term of $I_n(f)$, denoted by $E_n(f)$
\[
E_n(f):=\int_a^b \frac{f^{(n+1)}(\xi(x))\psi(x)}{(n+1)!}\diff x = \frac{1}{(n+1)!}\int_a^b f^{(n+1)}(\xi(x))\psi(x)\diff x
\]
where $\xi(x)$ is in $[a,b]$ for each $x$. \\Therefore we can write the integral of $f(x)$ as
\[
I(f):=\int_a^b f(x)\diff x = I_n(f)+E_n(f)
\]
where $I(f)\approx I_n(f)$.
\subsection{Trapezoidal Rule}
Trapezoidal rule is produced by let $n=1$, $x_0 = a$ and $x_1=b$. \\Under these condition, interpolating polynomial is
\[
p_1(x) = \frac{x-b}{a-b}f(a)+\frac{x-a}{b-a}f(b)
\]
a line passes through $(a,f(a))$ and $(b,f(b))$.\\ Then,
\begin{align*}
I(f)=\int_a^bf(x)\diff x &= \int_a^b p_1(x)\diff x +E_1(f)\\
&\approx \frac{b-a}{2}[f(a)+f(b)]\;\;\;(\text{Trapezoidal Rule})
\end{align*}
by ignoring error term $E_1(f)$.\\The error term is
\begin{align*}
E_1(f)&=\int_a^b\frac{1}{2}f^{\prime\prime}(\xi(x))(x-a)(x-b)\diff x\\
&= \frac{1}{2}f^{\prime\prime}(\xi)\int_a^b (x-a)(x-b)\diff x\;\;\;(\text{by MVT for integrals})\\
&=\frac{1}{2}f^{\prime\prime}(\xi)\left(-\frac{(b-a)^3}{6}\right)\\
&=-\frac{1}{12}f^{\prime\prime}(\xi)
\end{align*}
\begin{definition}[Degree of Precision]
\hfill\\\normalfont A quadrature formula $I_n(f)$ that approximates $I(f)$ has degree of precision $m$ if it is exact for all polynomial $f$ of degree $\leq m$, and $I_n(f)\neq I(f)$ for some polynomial $f$ of degree $m+1$.
\end{definition}
\begin{theorem}[Equivalent Definition of degree of precision]
\hfill\\\normalfont The following are equivalent.
\begin{enumerate}
  \item $I_n(f)$ has degree of precision $m$.
  \item $\int_a^b f(x)\diff x -I_n(f)=0\;\;\forall f(x)=p_m(x)$ and
  $\int_a^b f(x)\diff x -I_n(f)\neq 0\;\;\text{for some }f(x)=p_{m+1}(x)$
  \item $\int_a^b x^i\diff x-I_n(x^i)=0\;\;\forall i = 0,1,\ldots, m$ and
  $\int_a^b x^{m+1}\diff x -I_n(x^{m+1})\neq 0$ 
\end{enumerate}
\end{theorem}
\subsection{Simpson's Rule}
Using the similar technique, by setting $n=2$, a three equidistant point $x_0=a, x_1=\frac{1}{2}(a+b)$ and $x_2=b$, we have
\begin{align*}
p_2(x)&=\frac{(x-x_1)(x-x_2)}{(x_0-x_1)(x_0-x_2)}f(x_0)+\frac{(x-x_0)(x-x_2)}{(x_1-x_0)(x_1-x_2)}f(x_1)+\frac{(x-x_0)(x-x_1)}{(x_2-x_0)(x_2-x_1)}f(x_2)\\
&=\frac{1}{2h^2}[(x-x_1)(x-x_2)f(x_0)-2(x-x_0)(x-x_2)f(x_1)+(x-x_0)(x-x_1)f(x_2)]
\end{align*}
where $h=\frac{1}{2}(b-a)=x_{i+1}-x_i, i=0,1$ is the step size, and
\[
E_2(f)=\frac{f^{\prime\prime\prime}(\xi)}{3!}\psi(x)=(x-x_0)(x-x_1)(x-x_2)f[x,x_0,x_1,x_2]
\]
It can be shown that $I_2(f)$ is
\[
I_2(f)=\frac{h}{3}[f(x_0)+4f(x_1)+f(x_2)]
\]
and
\[
E_2(f)=-\frac{h^5}{90}f^{(4)}(\xi)
\]
for some $\xi\in[x_0,x_2]$.\\
It is then clear the degree of precision of Simpson's rule is $m=3$.
\subsubsection{Proof of Simpson's error formula}
let $F$ be an anti-derivative of $f$, i.e. $F^\prime = f$. WLOG, $a+b=0$, set $x=\frac{b-a}{2}$ then we are interested in the error expression
\[
E_4(f)=F(x)-F(-x)-\frac{x}{3}(f(x)+4f(0)+f(-x))
\]
This has derivatives
\begin{alignat}{2}
g'(x)&=\frac23(f(x)-2f(0)+f(-x))&&-\frac x3 (f'(x)-f'(-x))
\\
g''(x)&=\frac13(f'(x)-f'(-x))&&-\frac x3(f''(x)+f''(-x))
\\
g'''(x)&=&&-\frac x3(f'''(x)-f'''(-x))\;\;\;(\#).
\end{alignat}
Consequently, as $g(0)=g^\prime(0)=g''(0)=g'''(0)=0$, we have, by extended mean value theorem\footnote{Extended mean value theorem: }
\[
\frac{g(x)}{x^5}=\frac{g'(x_1)}{5x_1^4}=\cdots =\frac{g'''(x_3)}{(5\cdot4\cdot3x_3^2)}
\]
By substituting $(\#)$, we have
\begin{align*}
\frac{g(x)}{x^5}=\frac{-\frac{x_3}{3}(f'''(x_3)-f'''(-x_3))}{60x_3^2}&=-\frac{1}{90}\cdot\frac{f'''(x_3)-f'''(-x_3)}{2x_3}\\
&=\frac{1}{90}\cdot f^{(4)}(x_4)\;\;\;\text{By MVT}
\end{align*}

Other error bounds are
\begin{itemize}
  \item $|E_2(f)|\leq \frac{h^5}{12}|f^{(4)}(\xi)|$. This is derived by $E_2(f)=\int_{x_0}^{x_1} E_2(x)\diff x+\int_{x_1}^{x_2}E_2(x)\diff x$ and apply integral MVT on each integral.\footnote{Integral of $\int_{x_0}^{x_n}(x-x_0)\cdot(x-x_n)\diff x$ can be more easily evaluated by substitution: $x=x_0+th$, where $h=x_{i+1}-x_i$ is the step size. Another useful substitution is to make the above integrand symmetric.}
  \item $|E_2(f)|=-\frac{h^5}{12}[\frac{1}{3}f^{(4)}(\eta_1)-\frac{1}{5}f^{(4)}(\eta_2)]$. This is proved by (1) Taylor expansion of $f(x)$ around $x_1$ with remainder containing 4th power derivative. Odd power derivative are eliminated due to symmetry. (2) Next, consider Taylor expansion with remainder containing 4th power derivative around $x_0$ and $x_2$, both evaluated at $x_1$, to obtain an expression for $f''$ , as odd power derivatives are again eliminated. The fourth power derivative is bounded by IVT. Together the result is proved. 
\end{itemize}
\subsection{General Newton Cotes formula}
\begin{theorem}[Newton Cotes formula]
\hfill\\\normalfont Suppose that
\[
\sum_{i=0}^n a_if(x_i)
\]
denotes the $(n+1)$-point Newton-Cotes formula with
\[
x_0=a,\;\;x_n=b,\;\;h=\frac{b-a}{n} 
\]
There exists $\xi\in(a,b)$ such that
\[
\int_a^bf(x)\diff x = \sum_{i=0}^n a_if(x_i)+\frac{h^{n+3}f^{(n+2)}(\xi)}{(n+2)!}\int_0^nt^2(t-1)\cdot(t-n)\diff t
\]
if $n$ is \textbf{even} and $f\in C^{n+2}[a,b]$; and
\[
\int_a^bf(x)\diff x = \sum_{i=0}^n a_if(x_i)+\frac{h^{n+2}f^{(n+1)}(\xi)}{(n+1)!}\int_0^nt(t-1)\cdot(t-n)\diff t
\]
if $n$ is \textbf{odd} and $f\in C^{n+1}[a,b]$.\\
Therefore, when $n$ is an \textbf{even} integer, the degree of precision is $n+1$; when $n$ is \text{odd}, the degree of precision is only $n$.
\end{theorem}
\subsection{Composite Numerical Integration}
Composite numerical integration is performed in a piecewise approach.
\subsubsection{Composite Trapezoidal Rule}
Let
\[
h=\frac{b-a}{n},\;\;n\geq 1
\]
and
\[
x_i = a+ih,\;\; i=0,1,\ldots, n
\]
Then composite trapezoidal rule is stated as following
\begin{align*}
I(f)&=\int_a^b f(x)\diff x = \sum_{i=1}^n\int_{x_i-1}^{x_{i}}f(x)\diff x\\
&=\sum_{i=1}^n \left\{\frac{h}{2}[f(x_{i-1}+f(x_i)]-\frac{h^3}{12}f^{\prime\prime}(\xi_i)\right\}\\
&:=T_n(f)+E_{T_n}(f)
\end{align*}
for some $\xi_i\in[a,b]$. Denoting $f_i:=f(x_i)$, we have
\[
T_n(f=\frac{h}{2}[(f_0+f_1)+(f_1+f_2)+\cdots+(f_{n-1}+f_n)]=\frac{h}{2}[f_0+2\sum_{i=1}^{n-1}f_i+f_n]
\]
and
\begin{align*}
E_{t_n}(f)=I(f)-T_n(f)&=\sum_{i=1}^n-\frac{h^3}{12}f^{\prime\prime}(\xi_i)\\
&=-\frac{nh^3}{12}\left[\frac{1}{n}\sum_{i=1}^n f''(\xi_i)\right]\\
&=-\frac{b-a}{12}h^2\left[\frac{1}{n}\sum_{i=1}^n f''(\xi_i)\right]\\
&=-\frac{b-a}{12}h^2f''(\eta)\;\;\;\text{by Intermediate Value Theorem}
\end{align*}
From the error formula $E_{T_n}$, the composite trapezoidal rule has degree of precision 1.
\subsubsection{Composite Simpson's Rule}
For $n\geq 2$ and $n$ even, define $h=\frac{b-a}{n}$ and $x_i=a+ih, f_i=f(x_i)$ where $i=0,1,\ldots, n$. \\The composite Simpson's rule is stated as following
\begin{align*}
I(f)&=\int_{a}^bf(x)\diff x = \sum_{i=1}^\frac{n}{2}\int_{x_{2i-2}}^{x_{2i}}f(x)\diff x\\
&=\sum_{i=1}^\frac{n}{2}\left[\frac{h}{3}(f_{2i-2}+4f_{2i-1}+f_{2i})-\frac{h^5}{90}f^{(4)}(\xi_i)\right]
\end{align*}
where $\xi_i\in[x_{2i-2},x_{2i}]$.\\
It can be shown that the expression for composite Simpson's rule, $I_{S_n}(f)$, is
\[
I_{S_n}(f)=\frac{h}{3}(f_0+4f_1+2f_2+4f_3+2f_4+\cdots+2f_{n-2}+4f_{n-1}+f_n)
\]
and the error $E_{S_n}(f)$ is
\begin{align*}
E_{s_n}(f)&=I(f)-I_{S_n}(f)\\
&=\sum_{i=1}^\frac{n}{2}\left[-\frac{h^5}{90}f^{(4)}(\xi_i)\right]\\
&=-\frac{h^5}{90}\sum_{i=1}^\frac{n}{2}f^{(4)}(\xi_i)\\
&=-\frac{h^5}{90}\frac{n}{2}f^{(4)}(\eta)\;\;\;\text{by Intermediate Value Theorem}\\
&=-\frac{h^4(b-a)}{180}f^{(4)}(\eta)
\end{align*}
for some $\eta\in(a,b)$.
\subsection{Romberg Integration}
The Romberg Integration is carried out in a recursive manner. The base case $R_{k,1}$ is calculated first.\\
Let
\[
h_k=\frac{b-a}{2^{k-1}}\;\;\;k=1,2,\ldots
\]
which corresponds to $n_k=2^{k-1}$ subintervals.\\
We denote the Composite Trapezoida; rule with $h_k$ above by $R_{k,1}$, i.e.,
\[
R_{k,1}=\frac{h_k}{2}\sum_{i=1}^{2^{k-1}}[f(a+(i-1)h_k)+f(a+ih_k)]
\]
It turns out that we can reduce calculation by using $R_{k-1,1}$ in $R_{k}$:
\[
R_{k,1}=\frac{1}{2}[R_{k-1,1}+h_{k-1}\sum_{i=1}^{2^{k-2}}f(a+(2i-1)h_k)]
\]
It can be shown that
\[
\int_a^bf(x)\diff x -R_{k,1}=\sum_{i=1}^\infty K_ih_k^{2i}=K_1h_k^2+\sum_{i=2}^\infty K_ih_k^{2i}
\]
where $K_i (i=1,2,\ldots)$ are \textbf{constants} \textit{independent} of $h_k$ and they depend only on $f^{(2i-1)}(a)$ and $f^{(2i-1)}(b)$.

With this result, we can accelerate the covergence of the Composite Trapezoidal rule $R_{k,1}$, by eliminating the terms involving $h_k^2$, using expression $R_{k,1}$ and $R_{k+1,1}$. It can be shown that
\[
R_{k+1,2}=\frac{4R_{k+1,1}-R_{k,1}}{3}
\]
and 
\[
\int_a^bf(x)\diff x = R_{k+1,2}+O(h_k^4)
\]
whose degree of precision improves.\\
In general, we have
\[
R_{k,j}=\frac{4^{j-1}R_{k,j-1}-R_{k-1,j-1}}{4^{j-1}-1}
\]
and
\[
\int_a^bf(x)\diff x = R_{k,j}+O(h_k^{2j})
\]
In general, if we want to calculate $R_{n,n}$, we follow the following table
\[
\begin{bmatrix}
R_{1,1}&&&\\
R_{2,1}&R_{2,2}&&\\
\vdots&\vdots&\ddots&\\
R_{n,1}&R_{n,2}&\cdots&R_{n,n}
\end{bmatrix}
\]
and compute column by column.
\subsection{Adaptive Quadrature Methods}
Adaptive Quadrature is based on Simpson's rule. The objective is to bound the error of approximation to $\int_a^bf(x)\diff x$ by a specified error $\varepsilon>0$.\\
The first step is to apply Simpson's rule with step size $h=\frac{b-a}{2}$, producing
\[
\int_a^bf(x)\diff x = S(a,b)-\frac{h^5}{90}f^{(4)}(\mu)\;\;\;\text{for some }\mu\in(a,b)
\]
where $S(a,b)= \frac{h}{3}[f(a)+4f(a+h)+f(b)]$.
The next step is to apply composite Simpson's rule with $n=4$, and step size $\frac{h}{2}=\frac{b-a}{4}$,giving
\[
\int_a^bf(x)\diff x = S(a,\frac{a+b}{2})+S(\frac{a+b}{2},b)-\frac{1}{16}\frac{h^5}{90}f^{(4)}(\mu^\prime)
\]
By assumption $\mu=\mu^\prime$, we have
\[
\frac{h^5}{90}f^{(4)}(\mu)=\frac{16}{15}[S(a,b)-S(a,\frac{a+b}{2})-S(\frac{a+b}{2},b)]
\]
and this estimation of error gives
\[
\left|\int_a^bf(x)\diff x - S(a,\frac{a+b}{2})-S(\frac{a+b}{2},b)\right|\approx \frac{1}{15}|S(a,b)- S(a,\frac{a+b}{2})-S(\frac{a+b}{2},b)|
\]
Therefore, we check whether $\text{RHS}<\varepsilon$. If yes, stop.\\
If not, distribute the error into two subintervals, and try bound the error, in this case $\frac{1}{2}\varepsilon$ using the same technique. Repeat this step until the error is contained.
\subsection{Gaussian Quadrature}
Gaussian quadrature is of the form of $\sum_{i=1}^n w_if(x_i)$, where $x_i, i=1,2,\ldots, n$ are chosen from interval $[a,b]$ and $w_i, i=1,\ldots, n$ are chosen to minimise the expected error by maximising the degree of precision.\\
To obtain Gaussian quadrature of degree $n$ on interval $[-1,1]$ , we solve the equation
\[
E_n(x^i)=I(x^i)-I_n(x^i),\;\;\; i = 0,1,\ldots, 2n-1
\]
\textbf{Case 1}: $n=1$.\\
Then $I_1(f) = 2f(0)$.\\
\textbf{Case 2}: $n=2$.\\
Then $I_2(f) = f(\frac{-1}{\sqrt{3}})+f(\frac{1}{\sqrt{3}})$\\
\textbf{Case 3}: $n=3$.\\
Then $I_3(f) = \frac{5}{9}f(-\sqrt{\frac{3}{5}})+\frac{8}{9}f(0)+\frac{5}{9}f(\sqrt{\frac{3}{5}})$.\\
\textbf{Case $n$}:\\
Solve
\[
\int_{-1}^1x^j\diff x = \sum_{i=1}^n w_ix_i^j\;\;\;j=0,1,\ldots, 2n-1
\]
for $w_i, x_i, i = 1,2,\ldots, n$.

For general interval $[a,b]$, introduce linear change of variable
\[
x=\frac{(b+a)+t(b-a)}{2}\;\;\;-1\leq t\leq 1
\]
Thus,
\[
I(f) =\frac{b-a}{2}\int_{-1}^1f\left(\frac{(b+a)+t(b-a)}{2}\right)\diff t = \frac{b-a}{2}\int_{-1}^1g(t)\diff t
\]
where $g(t)=f\left(\frac{(b+a)+t(b-a)}{2}\right)$.
\clearpage
\section{Solutions of Nonlinear Equations}
There are three numerical methods for finding the root of $f(x)=0$, namely
\begin{itemize}
  \item Bisection Method
  \item Fixed Point Method
  \item Newton-Rhapson Method
\end{itemize}
\subsection{Bisection Method}
Intermediate value theorem is an important theorem used in bisection method, to demonstrate the existence of root.
\begin{theorem}[Intermediate value theorem]
\hfill\\\normalfont If the function $f(x)$ is defined and continuous on the real interval $[a,b]$ and if
\[
f(a)f(b)<0
\]
then there exists at least one real number $x^\ast$ such that
\[
f(x^\ast)=0\;\;\text{and}\;\;a<x^\ast<b
\]
\end{theorem}
The uniqueness of the root can be guaranteed if we have 
\[
f^\prime(x)>0\;\;\text{or}\;\;f^\prime(x)<0\;\;\text{on}\;[a,b]
\]
provided that $f^\prime(x)$ exists on $[a,b]$.
\begin{theorem}[Algorithm: Bisection Method]
\hfill\\\normalfont Let $f(x)$ be continuous on the interval $[a,b]$.\\Suppose $f(a_0)f(b_0)<0$, then for $k=1,2,\ldots$, compute
\[
x_k=\frac{1}{2}(a_{k-1}+b_{k-1})
\]
and the new interval $I_k$ is
\[
I_k=(a_k,b_k)=\begin{cases}
(x_k,b_{k-1})&\text{ if }f(x_k)f(b_{k-1})<0\\
(a_{k-1},x_k)&\text{ if }f(x_k)f(a_{k-1})<0
\end{cases}
\]
Note that the solution lies in $I_k=(a_k,b_k)$.\\
If $f(x_k)=0$, then we have computed a solution. If $f(x_k)\neq 0$ we can continue till $b_k-a_k\leq$ a specified tolerance and $x_k$ will be an approximate solution.
\end{theorem}
\begin{theorem}[Error of Bisection Method]
\hfill\\\normalfont From the estimate of solution after the $k$th bisection,
\[
x_k=\frac{1}{2}(a_{k-1}+b_{k-1})
\]
Since the exact solution is bounded in the interval $[a_{k-1}, b_{k-1}]$, the error estimates for $x_k$ the approximated solution to $f(x)=0$, is
\[
|x^\ast-x_k|\leq \frac{b_{k-1}-a_{k-1}}{2}=\frac{b_0-a_0}{2^k}
\]
where $x^\ast$ is the exact solution to $f(x)=0$.
\end{theorem}
\textbf{Remark}:
\begin{itemize}
  \item Error of bisection method will always converges to $0$.
  \item The rate of convergence is independent of the function $f(x)$.
  \item Error of estimate $x_k$ is said to be within $t$ correct places, if $|x^\ast-x_k|<\frac{1}{2}\times 10^{-t}$ is satisfied.
\end{itemize}
\subsection{Fixed Point Method}
\begin{definition}[Fixed Point]
\hfill\\\normalfont Consider the equation $g(x)=x$ where $g$ is the given function and $x$ is in some finite interval.\\A solution $\eta$ satisfies 
\[
g(\eta)=\eta
\]
is called a \textbf{fixed point} of $g$.
\end{definition}
\begin{theorem}[Existence and Uniqueness of Fixed Point]
\hfill\\\normalfont If $g$ is continuous on $[a,b]$ and $a\leq g(x)\leq b$ for all $x\in[a,b]$, then $g$ has a fixed point $\eta\in[a,b]$.\\
If, in addition, $g^\prime(x)$ exists on $(a,b)$ and
\[
|g^\prime(x)|\leq \alpha < 1,\;\;\forall x\in(a,b) \;\;\;(\#)
\]
then $g$ has a \textbf{unique} fixed point $\eta\in[a,b]$.
\end{theorem}
\begin{theorem}[Error Bound]
\hfill\\\normalfont Suppose $(\#)$ holds, an error bound is
\[
|x_n-\eta|\leq \alpha^n|x_0-\eta|
\]
A computable error bound is
\[
|x_0-\eta|\leq \frac{\alpha^n}{1-\alpha}|x_0-x_1|
\]
\end{theorem}
\textbf{Remark}: The important step is to find a qualifying $g$ before applying fixed point method.
\subsection{Order of Convergence}
\begin{definition}[Order of Convergence]
\hfill\\\normalfont In general, a sequence $\{x_n\}_{n=1}^\infty$ is said to \textbf{converge with order} $k$ to the limit $\eta$ if there is a constant $\lambda\neq 0$ such that
\[
\lim_{n\to\infty}\frac{|x_{n+1}-\eta|}{|x_n-\eta|^k}=\lambda
\]
\end{definition}
An iterative method
\[
x_{n+1}=g(x_n)\;\;\;n=0,1,\ldots
\]
is said to be of \textbf{order} $k$ if $\{x_n\}$ converges with order $k$ to the fixed point of $g$.
\begin{theorem}[Degree of Convergence]
\hfill\\\normalfont Suppose that the iterative method
\[
x_{n+1}=g(x_n)\;\;\;n=0,1,\ldots
\]
converges to a fixed point $\eta$ of $g$.\\
If $g^\prime(\eta)=g^{\prime\prime}(\eta)=\cdots=g^{(k-1)}(\eta)=0$ and $g^{(k)}(\eta)\neq 0$, and $g^{(k)}$ is continuous on an interval containing $\eta$, then
\[
\lim_{n\to\infty}\frac{|x_{n+1}-\eta|}{|x_n-\eta|^k}\neq 0
\]
that is, the convergence $x_n\to \eta$ is of order $k$.
\end{theorem}
\end{document}